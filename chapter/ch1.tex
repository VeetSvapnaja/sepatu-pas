\chapter{Ketika Sepatu Pas}

\bahasa
11 Oktober 1974 pagi di Aula Buddha

\english
11 October 1974 am in Buddha Hall

\bahasa
Chu'i si juru gambar dapat menggambar lingkaran lebih sempurna dengan tangan hampa daripada dengan sebuah kompas.

Jemarinya melahirkan bentuk spontan entah dari mana. Pikirannya keji selagi bebas dan tanpa mempedulikan apa yang sedang dilakukannya.

Tidak ada latihan yang dibutuhkan, pikirannya sangat sederhana dan tidak mengenal hambatan.

Jadi, saat sepatunya pas, kaki dilupakan, saat sabuk pas, perut dilupakan, saat hati benar, 'untuk' dan 'terhadap' dilupakan.

Tidak ada penggerak, tidak ada dorongan, tidak ada kebutuhan, tidak ada tarikan: maka usahamu terkendali. Engkau adalah orang yang bebas.

Santai adalah benar. Mulailah dengan benar dan engkau santai. Lanjutkan dengan santai dan engkau benar. Cara yang tepat untuk menjadi santai adalah melupakan jalan yang benar dan melupakan bahwa jalannya santai.

\english
CHU'I THE DRAFTSMAN COULD DRAW MORE PERFECT CIRCLES FREEHAND THAN WITH A COMPASS.

HIS FINGERS BROUGHT FORTH SPONTANEOUS FORMS FROM NOWHERE. HIS MIND WAS MEANWHILE FREE AND WITHOUT CONCERN WITH WHAT HE WAS DOING.

NO APPLICATION WAS NEEDED, HIS MIND WAS PERFECTLY SIMPLE AND KNEW NO OBSTACLE.

SO, WHEN THE SHOE FITS, THE FOOT IS FORGOTTEN, WHEN THE BELT FITS, THE BELLY IS FORGOTTEN, WHEN THE HEART IS RIGHT, ’FOR’ AND ’AGAINST’ ARE FORGOTTEN.

NO DRIVES, NO COMPULSIONS, NO NEEDS, NO ATTRACTIONS: THEN YOUR AFFAIRS ARE UNDER CONTROL. YOU ARE A FREE MAN.

EASY IS RIGHT. BEGIN RIGHT AND YOU ARE EASY. CONTINUE EASY AND YOU ARE RIGHT. THE RIGHT WAY TO GO EASY IS TO FORGET THE RIGHT WAY AND FORGET THAT THE GOING IS EASY.

\bahasa
Chuang Tzu adalah salah satu bunga yang langka, langka bahkan dari Buddha atau Yesus. Karena Buddha dan Yesus menekankan usaha dan Chuang Tzu menekankan tanpa usaha.

\english
Chuang Tzu is one of the rarest of flowerings, rarer even than a Buddha or a Jesus. Because Buddha and Jesus emphasise effort and Chuang Tzu emphasises effortlessness.


