\chapter{Ketika Sepatu Pas}

\bahasa
11 Oktober 1974 pagi di Aula Buddha

\english
11 October 1974 am in Buddha Hall

\bahasa
CHU'I SI JURU GAMBAR DAPAT MENGGAMBAR LINGKARAN LEBIH SEMPURNA DENGAN TANGAN HAMPA DARIPADA DENGAN SEBUAH KOMPAS.

\english
CHU'I THE DRAFTSMAN COULD DRAW MORE PERFECT CIRCLES FREEHAND THAN WITH A COMPASS.

\bahasa
JEMARINYA MELAHIRKAN BENTUK SPONTAN ENTAH DARI MANA. PIKIRANNYA SEMENTARA ITU BEBAS DAN TANPA MEMPEDULIKAN APA YANG SEDANG DILAKUKANNYA.

\english
HIS FINGERS BROUGHT FORTH SPONTANEOUS FORMS FROM NOWHERE. HIS MIND WAS MEANWHILE FREE AND WITHOUT CONCERN WITH WHAT HE WAS DOING.

\bahasa
TIDAK ADA LATIHAN YANG DIBUTUHKAN, PIKIRANNYA SANGAT SEDERHANA DAN TIDAK MENGENAL HAMBATAN.

\english
NO APPLICATION WAS NEEDED, HIS MIND WAS PERFECTLY SIMPLE AND KNEW NO OBSTACLE.

\bahasa
JADI, SAAT SEPATUNYA PAS, KAKI DILUPAKAN, SAAT SABUK PAS, PERUT DILUPAKAN, SAAT HATI BENAR, 'UNTUK' DAN 'TERHADAP' DILUPAKAN.

\english
SO, WHEN THE SHOE FITS, THE FOOT IS FORGOTTEN, WHEN THE BELT FITS, THE BELLY IS FORGOTTEN, WHEN THE HEART IS RIGHT, 'FOR' AND 'AGAINST' ARE FORGOTTEN.

\bahasa
TIDAK ADA PENGGERAK, TIDAK ADA DORONGAN, TIDAK ADA KEBUTUHAN, TIDAK ADA TARIKAN: MAKA USAHAMU TERKENDALI. ENGKAU ADALAH ORANG YANG BEBAS.

\english
NO DRIVES, NO COMPULSIONS, NO NEEDS, NO ATTRACTIONS: THEN YOUR AFFAIRS ARE UNDER CONTROL. YOU ARE A FREE MAN.

\bahasa
SANTAI ADALAH MUDAH. MULAILAH DENGAN MUDAH DAN ENGKAU SANTAI. LANJUTKAN DENGAN SANTAI DAN ENGKAU MUDAH. CARA YANG TEPAT UNTUK MENJADI SANTAI ADALAH MELUPAKAN JALAN YANG MUDAH DAN MELUPAKAN BAHWA JALANNYA ADALAH SANTAI.

\english
EASY IS RIGHT. BEGIN RIGHT AND YOU ARE EASY. CONTINUE EASY AND YOU ARE RIGHT. THE RIGHT WAY TO GO EASY IS TO FORGET THE RIGHT WAY AND FORGET THAT THE GOING IS EASY.

\bahasa
Chuang Tzu adalah salah satu bunga yang langka, lebih langka bahkan daripada seorang Buddha atau seorang Yesus. Karena Buddha dan Yesus menekankan usaha dan Chuang Tzu menekankan tanpa usaha.

\english
Chuang Tzu is one of the rarest of flowerings, rarer even than a Buddha or a Jesus. Because Buddha and Jesus emphasise effort and Chuang Tzu emphasises effortlessness.

\bahasa
Banyak yang dapat dilakukan melalui usaha tapi lebih banyak dapat dilakukan melalui tanpa usaha. Banyak yang dapat dicapai melalui kehendak tapi banyak lagi dapat dicapai melalui tanpa kehendak. Dan apa pun yang engkau capai melalui kehendak akan selalu menjadi beban bagimu; itu akan selalu menjadi sebuah konflik, sebuah ketegangan batin, dan engkau dapat kehilangannya kapanpun. Hal itu harus terus dipertahankan - dan mempertahankannya membutuhkan energi, mempertahankannya akhirnya mencerai-beraikan dirimu.

\english
Much can be done through effort but more can be done through effortlessness. Much can be achieved through will but much more can be achieved through will-lessness. And whatsoever you achieve through will will always remain a burden to you; it will always be a conflict, an inner tension, and you can lose it at any moment.It has to be maintained continuously - and maintaining it takes energy, maintaining it finally dissipates you.

\bahasa
Hanya apa yang dicapai melalui tanpa-usaha tidak akan pernah menjadi sebuah beban bagimu, dan hanya apa yang tidak menjadi sebuah beban dapat abadi. Hanya apa yang tidak dengan cara apa pun yang tidak wajar dapat tetap bersamamu selamanya dan selamanya.

\english
Only that which is attained through effortlessness will never be a burden to you, and only that which is not a burden can be eternal. Only that which is not in any way unnatural can remain with you forever and forever.

\bahasa
Chuang Tzu mengatakan bahwa yang sebenarnya, yang ilahi, eksistensial, dicapai dengan kehilangan diri sepenuhnya di dalamnya. Bahkan upaya untuk mencapainya menjadi sebuah penghalang - maka engkau tidak dapat kehilangan diri sendiri. Bahkan usaha untuk kehilangan diri menjadi sebuah penghalang.

\english
Chuang Tzu says that the real, the divine, the existential, is to be attained by losing yourself completely in it. Even the effort to attain it becomes a barrier - then you cannot lose yourself. Even the effort to lose yourself becomes a barrier.

\bahasa
Bagaimana engkau dapat membuat usaha untuk kehilangan diri sendiri? Semua usaha lahir dari ego, dan melalui usaha ego diperkuat. Ego adalah penyakitnya. Jadi semua usaha harus ditinggalkan sama sekali, tidak ada yang harus dilakukan; kita harus kehilangan diri sepenuhnya dalam eksistensial. Kita harus menjadi seperti anak kecil, baru lahir, tidak tahu apa yang benar, tidak tahu apa yang salah, tidak mengetahui perbedaan apapun. Begitu perbedaan masuk, setelah engkau tahu ini benar dan itu salah, engkau sudah sakit, dan engkau jauh dari kenyataan.

\english
How can you make any effort to lose yourself? All effort is born out of the ego, and through effort ego is strengthened. Ego is the disease. So all effort has to be left completely, nothing is to be done; one has to lose oneself completely in the existential. One has to become again like a small child, just born, not knowing what is right, not knowing what is wrong, not knowing any distinctions. Once distinctions enter, once you know this is right and that is wrong, you are already ill, and you are far away from reality.

\bahasa
Seorang anak hidup secara alami - dia total. Dia tidak membuat usaha apapun, karena melakukan usaha berarti engkau berkelahi dengan diri sendiri. Sebagian dari dirimu adalah untuk dan sebagian dari dirimu melawan - maka dari itu adanya upaya.

\english
A child lives naturally - he is total. He does not make any effort, because making an effort means you are fighting with yourself. A part of you is for and a part of you is against - hence the effort.

\bahasa
Engkau dapat meraih banyak, ingatlah. Di dunia ini, khususnya, engkau dapat meraih banyak melalui usaha karena usaha adalah agresi, usaha adalah kekerasan, usaha adalah persaingan. Tapi di dunia lain tidak ada yang dapat dicapai melalui usaha, dan mereka yang memulai dengan usaha akhirnya juga harus menjatuhkannya.

\english
You can achieve much, remember. In this world, particularly, you can achieve much through effort because effort is aggression, effort is violence, effort is competition. But in the other world nothing can be achieved through effort, and those who start with effort finally have also to drop it.

\bahasa
Buddha bekerja selama enam tahun, terus bermeditasi, berkonsentrasi - dia menjadi seorang pertapa. Dia melakukan semua yang dapat dilakukan oleh manusia, tidak ada satu pun batu yang terlewat - dia mempertaruhkan seluruh keberadaannya. Tapi itu adalah usaha, ego ada di sana, dan dia gagal.

\english
Buddha worked for six years, continuously meditating, concentrating - he became an ascetic. He did all that can be done by a human being, not a single stone was left unturned - he staked his whole being. But it was an effort, the ego was there, and he failed.

\bahasa
Tidak ada yang gagal seperti ego di yang terutama; tidak ada yang berhasil seperti ego di dunia ini.

\english
Nothing fails like the ego in the Ultimate; nothing succeeds like the ego in this world.

\bahasa
Dalam dunia materi tidak ada yang berhasil seperti ego; dalam dunia kesadaran tidak ada yang gagal seperti ego. Kasusnya justru berlawanan - dan memang harus seperti itu karena dimensinya justru berlawanan.

\english
In the world of matter nothing succeeds like the ego; in the world of consciousness nothing fails like the ego. The case is just the opposite - and it has to be so because the dimension is just the opposite.

\bahasa
Buddha gagal total. Setelah enam tahun dia benar-benar frustrasi, dan ketika aku mengatakan benar-benar, maksud ku benar-benar sepenuhnya. Tak ada satu pun serpihan harapan yang tersisa, dia menjadi sangat tidak berdaya. Dalam tanpa daya itu, dia menjatuhkan semua usaha. Dia sudah menjatuhkan dunia, dia sudah meninggalkan kerajaannya; semua milik dunia yang terlihat ini, dia telah meninggalkannya, melepaskan semuanya.

\english
Buddha failed absolutely. After six years he was completely frustrated, and when I say completely, I mean completely. Not even a single fragment of hope remained, he became absolutely hopeless. In that hopelessness he dropped all effort. He had already dropped the world, he had already left his kingdom; all that belonged to this visible world, he had left, renounced.

\bahasa
Sekarang setelah enam tahun usaha keras ia meninggalkan juga semua milik dunia lain. Dia benar-benar hampa - kosong. Malam itu ia tidur dengan kualitas tidur yang berbeda karena tidak ada ego; kualitas keheningan yang berbeda muncul karena tidak ada usaha; kualitas yang berbeda terjadi padanya malam itu karena tidak ada mimpi.

\english
Now after six years of strenuous effort he left also all that belonged to the other world. He was in a complete vacuum - empty. That night he slept a different quality of sleep because there was no ego; a different quality of silence arose because there was no effort; a different quality of being happened to him that night because there was no dreaming.

\bahasa
Jika tidak ada usaha, tidak ada yang tidak lengkap, maka tidak perlu bermimpi. Mimpi selalu menyelesaikan sesuatu: sesuatu yang tidak lengkap pada siang hari akan selesai dalam mimpi karena pikiran memiliki kecenderungan untuk menyelesaikan semuanya. Jika tidak lengkap maka pikiran akan selalu tidak nyaman. Usaha dimasukkan ke dalam banyak hal dan jika hal-hal itu tetap tidak lengkap, sebuah mimpi dibutuhkan.

\english
If there is no effort, nothing is incomplete, then there is no need to dream. A dream is always to complete something: something which has remained incomplete in the day will be completed in a dream because mind has a tendency to complete everything. If it is not complete then the mind will always be uneasy. Effort is put into many things and if they remain incomplete, a dream is needed.

\bahasa
Bila ada keinginan, pasti ada mimpi, karena menginginkan adalah bermimpi - bermimpi hanyalah bayangan keinginan.

\english
When there is desire, there is bound to be dreaming, because desiring is dreaming - dreaming is just a shadow of desiring.

\bahasa
Malam itu, ketika tidak ada yang dapat dilakukan - dunia ini sudah tidak ada gunanya, sekarang dunia lain juga tidak berguna - semua motivasi untuk bergerak berhenti. Tidak ada tempat untuk pergi, dan tidak ada seseorang yang dapat pergi ke mana pun. Tidur malam itu menjadi samadhi, itu menjadi satori; hal itu menjadi hal paling utama yang dapat terjadi pada seseorang. Buddha berbunga malam itu dan di pagi hari dia tercerahkan. Dia membuka matanya, melihat bintang terakhir menghilang di langit, dan semuanya ada di sana. Itu selalu ada di sana, tapi dia sangat menginginkannya sehingga dia tidak dapat melihatnya. Itu selalu ada di sana; tapi dia telah bergerak jauh di masa depan dengan keinginan sehingga dia tidak dapat melihat pada disini dan sekarang.

\english
That night, when there was nothing to be done - this world was already useless, now the other world was also useless - all motivation to move ceased. There was nowhere to go, and there was no one to go anywhere That night sleep became samadhi, it became satori; it became the ultimate thing that can happen to a man. Buddha flowered that night and in the morning he was enlightened. He opened his eyes, looked at the last star disappearing in the sky, and everything was there. It had always been there, but he had wanted it so much he couldn't see it. It had always been there; but he had been moving so much in the future with the desire that he could not look at the here and now.

\bahasa
Malam itu tidak ada keinginan, tidak ada tujuan, tidak ada tempat untuk pergi, dan tidak ada seseorang yang dapat pergi ke mana pun - semua usaha berhenti. Tiba-tiba dia menjadi sadar akan dirinya sendiri, tiba-tiba dia menyadari kenyataan seperti apa adanya. Chuang Tzu mengatakan sejak awal: Jangan melakukan usaha apapun. Dan dia benar. Karena engkau tidak akan pernah membuat usaha total seperti Buddha. Engkau tidak akan pernah merasa frustrasi sehingga usaha itu jatuh dengan sendirinya; itu akan selalu tidak lengkap. Dan pikiranmu akan selalu terus berkata: Sedikit lagi dan sesuatu akan terjadi, selangkah lagi. Tujuannya sudah dekat, mengapa engkau merasa berkecil hati? Hanya sedikit lagi usaha diperlukan karena tujuan semakin dekat setiap hari.

\english
That night there was no desire, no goal, nowhere to go, and no one to go anywhere - all effort ceased. Suddenly he became aware of himself, suddenly he became aware of the reality as it is. Chuang Tzu says from the very beginning: Don't make any effort. And he is right. Because you will never make such a total effort as Buddha. You will never be so frustrated that the effort drops by itself; it will always be incomplete. And your mind will always go on saying: A little more and something will happen, a step more. The goal is near, why are you getting dejected? Just a little more effort is needed because the goal is coming nearer every day.

\bahasa
Karena engkau tidak akan pernah berusaha sekuat tenaga sehingga engkau tidak akan pernah benar-benar putus asa. Dan engkau dapat melanjutkan usaha setengah hati ini untuk banyak kehidupan - itulah yang telah engkau lakukan di masa lalu. Engkau tidak berada di sini untuk pertama kalinya dihadapanku. Engkau tidak di sini untuk pertama kalinya membuat beberapa upaya untuk menyadari yang benar, yang sebenarnya. Engkau telah berkali-kali melakukannya, berkali-kali, satu juta kali di masa lalu - tapi engkau masih berharap.

\english
Because you will never make so absolute an effort you will never be completely hopeless. And you can continue this half-hearted effort for many lives - that is what you have been doing in the past. You are not here for the first time before me. You are not here for the first time making some effort to realise the true, the real. You have done it many, many times, a million times in the past - but you are still hopeful.

\bahasa
Chuang Tzu mengatakan; lebih baik menjatuhkan usaha di awal. Itu harus dijatuhkan: entah engkau menjatuhkannya di awal atau engkau harus menjatuhkannya di akhir. Tapi akhirnya mungkin tidak segera datang! Jadi ada dua cara: baik melakukan usaha total ... jadi total sehingga semua harapan hancur dan engkau menyadari bahwa tidak ada yang dapat dicapai melalui usaha, bahkan tidak ada satu serpihan kecil di suatu tempat di alam bawah sadar yang masih tersisa dan mengatakan : Lakukan sedikit lagi dan ini akan tercapai .... baik itu membuat usaha total, maka itu jatuh dengan sendirinya, atau sama sekali tidak berusaha. Hanya pahami semuanya. Jangan pindah ke sana bagaimanapun juga. Ingat satu hal ... engkau tidak dapat keluar dari situ jika tidak lengkap; begitu masuk, itu harus selesai. Karena pikiran memiliki kecenderungan untuk menyelesaikan segalanya - tidak hanya pikiran manusia, bahkan pikiran binatang. Jika engkau menggambar setengah lingkaran, tidak lengkap, dan seekor gorila datang dan melihatnya, dan jika ada kapur tulis di sana, dia akan segera menyelesaikannya.

\english
Chuang Tzu says; it is better to drop effort in the beginning. It has to be dropped: either you drop it in the beginning or you will have to drop it in the end. But the end may not come soon! So there are two ways: either make a total effort... so total that all hope is shattered and you come to realise that nothing can be achieved through effort, there is not even a single small fragment somewhere in the unconscious still lingering and saying: Do a little more and this will be achieved.... either make a total effort, then it drops by itself, or don't make any effort at all. Just understand the whole thing. Don't move into it at all. Remember one thing... you cannot come out of it if it is incomplete; once entered, it has to be completed. Because the mind has a tendency to complete everything - not only the human mind, even the animal mind. If you draw a half circle, incomplete, and a gorilla comes and sees it, and if some chalk is there, he will immediately complete it.

\bahasa
Pikiranmu memiliki kecenderungan untuk menyelesaikan - apapun yang tidak lengkap memberimu ketegangan. Jika engkau ingin tertawa dan engkau tidak bisa, akan ada ketegangan. Jika engkau ingin menangis dan tidak bisa, akan ada ketegangan. Jika engkau ingin marah dan tidak bisa, akan ada ketegangan. Itu sebabnya engkau telah sakit begitu lama; semuanya telah ditinggalkan tidak lengkap!

\english
Your mind as such has a tendency to complete - anything incomplete gives you tension. If you wanted to laugh and you could not, there will be tension. If you wanted to cry and could not, there will be tension. If you wanted to be angry and could not, there will be tension. That's why you have been ill for so long; everything has been left incomplete!

\bahasa
Engkau tidak pernah tertawa dengan total, engkau tidak pernah menangis dengan total, engkau sama sekali tidak pernah marah dengan total, engkau tidak pernah membenci dengan total, engkau tidak pernah mencintai dengan total. Tidak ada yang pernah dilakukan dengan total - semuanya tidak lengkap. Tidak ada yang total. Itu terus berlanjut, dan kemudian selalu ada banyak hal di dalam pikiranmu. Itulah mengapa engkau merasa tidak nyaman. Engkau tidak akan pernah merasa seperti di rumah sendiri.

\english
You have never laughed totally, you have never cried totally, you have never been angry totally, you have never hated totally, you have never loved totally. Nothing has been done totally - everything is incomplete. Nothing is total. It lingers on, and then there are always many things on your mind.That is why you are so ill at ease. You can never feel at home.

\bahasa
Chuang Tzu mengatakan: Lebih baik tidak memulai karena begitu engkau memulainya itu harus selesai. Pahami, dan jangan bergerak dalam lingkaran setan. Itulah sebabnya aku mengatakan bahwa Chuang Tzu adalah pembungaan yang langka, lebih langka dari Buddha atau Yesus. Karena dia mencapai hanya dengan pengertian. Tidak ada metode, tidak ada meditasi untuk Chuang Tzu. Dia berkata: Cukup mengerti 'fakta' itu. Engkau lahir. Usaha apa yang telah engkau buat untuk dilahirkan? Engkau tumbuh. Usaha apa yang telah engkau buat untuk tumbuh? Engkau bernafas. Usaha apa yang telah engkau lakukan untuk bernafas? Semuanya bergerak sendiri, jadi mengapa risau? Biarkan hidup mengalir sendiri maka engkau akan berada dalam keikhlasan. Jangan bergumul dan jangan mencoba bergerak ke hulu, jangan pernah mencoba berenang, hanya mengapung dengan arus dan membiarkan arus memimpinmu kemanapun ia mengarah. Jadilah awan putih yang bergerak di langit - tidak ada tujuan, tidak ke mana-mana, hanya mengambang. Yang mengambang adalah pembungaan yang terpokok.

\english
Chuang Tzu says: It is better not to start because once you start it has to be completed. Understand, and don't move in a vicious circle. That is why I say that Chuang Tzu is a rare flowering, rarer than a Buddha or a Jesus. Because he achieved simply by understanding. There is no method, no meditation for Chuang Tzu. He says: Simply understand the 'facticity' of it. You are born. What effort have you made to be born? You grow. What effort have you made to grow? You breathe.
What effort have you made to breathe? Everything moves on its own, so why bother? Let life flow on its own then you will be in a let go. Don't struggle and don't try to move upstream, don't even try to swim, just float with the current and let the current lead you wherever it leads. Be a white cloud moving in the sky - no goal, going nowhere, just floating. That floating is the ultimate flowering.

\bahasa
Jadi hal pertama yang harus dipahami tentang Chuang Tzu sebelum kita masuk ke sutranya, adalah - jadilah alami. Segala sesuatu yang tidak alami harus dihindari. Jangan melakukan apapun yang tidak wajar. Alam sudah cukup - engkau tidak dapat memperbaikinya, tapi ego mengatakan, tidak, engkau dapat memperbaiki alam - itulah bagaimana semua budaya ada.

\english
So the first thing to understand about Chuang Tzu before we enter his sutras, is - be natural. Everything unnatural has to be avoided. Don't do anything that is unnatural. Nature is enough - you cannot improve upon it, but the ego says, no, you can improve upon nature - that is how all culture exists.

\bahasa
Setiap usaha untuk memperbaiki alam adalah budaya, dan semua budaya itu seperti penyakit - semakin manusia itu berbudaya semakin ia berbahaya.

\english
Any effort to improve upon nature is culture, and all culture is like a disease - the more a man is cultured the more dangerous he is.

\bahasa
Suatu ketika seorang pemburu, seorang pemburu dari Eropa, hilang di hutan di Afrika. Tiba-tiba dia menemukan beberapa gubuk. Dia belum pernah mendengar bahwa ada desa di hutan lebat itu. Desa itu tidak ada di peta manapun. Jadi dia mendekati kepala desa, dan dia berkata: Sayang sekali engkau menghilang dari peradaban. Kepala suku berkata: Tidak, itu bukan sesuatu yang disayangkan, kami selalu takut ditemukan - begitu peradaban masuk, kami menghilang. Alam menghilang begitu engkau berusaha memperbaikinya - itu berarti engkau mencoba memperbaiki Tuhan. Semua agama mencoba melakukan itu - untuk memperbaiki Tuhan. Chuang Tzu tidak mendukung hal itu. Dia mengatakan alam adalah yang tertinggi, dan sifat tertinggi yang dia sebut Tao. Tao berarti bahwa alam adalah yang tertinggi dan tidak dapat diperbaiki. Jika engkau mencoba memperbaikinya, engkau akan melumpuhkannya - begitulah cara kita melumpuhkan setiap anak.

\english
I have heard that a hunter, a European hunter, was lost in a forest in Africa. Suddenly he came upon a few huts. He had never heard that a village existed in that thick forest. It was not on any map. So he approached the chief of the village, and he said: This is a pity that you are lost to civilisation. The chief said: No, it is not a pity, we are always afraid of being discovered - once civilisation comes in we are lost. Nature is lost once you make an effort to improve upon it - that means you are trying to improve upon God. All religions are trying to do that - to improve upon God. Chuang Tzu is not in favour of that. He says nature is ultimate, and that ultimate nature he calls Tao. Tao means that nature is ultimate and cannot be improved. If you try to improve upon it, you will cripple it - that is how we cripple every child.

\bahasa
Setiap anak lahir didalam Tao, kemudian kita melumpuhkannya dengan masyarakat, peradaban, budaya, moralitas, agama ... kita melumpuhkannya dari segala sisi. Lalu dia hidup, tapi dia tidak hidup.

\english
Every child is born in Tao, then we cripple him with society, civilisation, culture, morality, religion... we cripple him from every side. Then he lives, but he is not alive.

\bahasa
Suatu ketika seorang gadis kecil akan pergi ke sebuah pesta, pesta ulang tahun temannya. Dia sangat kecil, baru berumur empat tahun. Dia bertanya kepada ibunya: Apakah dulu ada pesta dan tarian seperti saat engkau masih hidup?

\english
I have heard that a small girl was going to a party, a friend's birthday party. She was very small, just four years old. She asked her mother: Were there such parties and dances when you were alive?

\bahasa
Semakin berbudaya dan beradab, semakin mati. Jika engkau ingin melihat orang-orang yang benar-benar mati tapi masih hidup pergilah ke biarawan di biara-biara, pergilah ke para imam di gereja-gereja, pergilah ke Paus di Vatikan. Mereka tidak hidup, mereka sangat takut akan kehidupan, begitu takut akan alam, sehingga mereka telah menekannya dari mana-mana. Mereka sudah berada di kuburan mereka. Engkau dapat melukis makamnya, engkau bahkan dapat membuat kuburan marmer, sangat berharga - tapi orang di dalamnya sudah mati.

\english
The more cultured and civilised, the more dead. If you want to see perfectly dead men and yet still alive go to the monk in the monasteries, go to the priests in the churches, go to the Pope in the Vatican. They are not alive, they are so afraid of life, so afraid of nature, that they have suppressed it from everywhere. They are already in their graves. You can paint the grave, you can even make a marble grave, very valuable - but the man inside is dead.

\bahasa
Seorang pemabuk sedang melewati sebuah kuburan dan dia melihat sebuah kuburan indah yang terbuat dari marmer putih murni. Dia melihat kuburan itu, melihat nama di atasnya. Makam itu milik Rothschild yang terkenal. Dia tertawa, dan berkata: Keluarga Rothschild ini, mereka tahu bagaimana cara hidup.

\english
A drunkard was passing through a graveyard and he saw a beautiful grave made of pure white marble. He looked at the grave, looked at the name on it. The grave was that of the famous Rothschild. He laughed, and said: These Rothschilds, they know how to live.

\bahasa
Kebudayaan membunuhmu, kebudayaan adalah pembunuh, kebudayaan adalah racun yang lambat - itu adalah bunuh diri.

\english
Culture kills you, culture is a murderer, culture is a slow poison - it is a suicide.

\bahasa
Chuang Tzu dan master tuanya, Lao Tzu, menentang budaya. Mereka mendukung alam, alam yang murni. Pohon berada dalam posisi yang lebih baik daripadamu, bahkan burung, ikan di sungai, berada dalam posisi yang lebih baik karena mereka lebih hidup, mereka lebih banyak menari dengan ritme alam. Engkau benar-benar lupa apa itu alam. Engkau telah mengutuknya sampai ke akar-akarnya

\english
Chuang Tzu and his old master, Lao Tzu, are against culture. They are for nature, pure nature. Trees are in a better position than you, even birds, fishes in the river, are in a better position because they are more alive, they dance more to the rhythm of nature. You have completely forgotten what nature is. You have condemned it to the very root.

\bahasa
Dan jika engkau ingin mengutuk alam engkau harus memulai dengan mengutuk seks, karena seluruh alam muncul darinya. Seluruh alam adalah luapan energi seks, cinta. Burung-burung bernyanyi, pohon berbunga - ini semua energi seksual, yang meledak. Bunga adalah simbol seks, nyanyian burung bersifat seksual, keseluruhan Tao tidak lain adalah energi seks - keseluruhan alam menyebar dengan sendirinya, mencintai dirinya sendiri, bergerak ke dalam kegembiraan cinta dan semesta yang lebih dalam.

\english
And if you want to condemn nature you have to start by condemning sex, because the whole of nature arises out of it. The whole of nature is an overflowing of sex energy, of love. The birds sing, the trees flower - this is all sexual energy, exploding. Flowers are sex symbols, the singing of the birds is sexual, the whole of Tao is nothing but sex energy - the whole of nature propagates itself, loves itself, moves into deeper ecstasies of love and existence.

\bahasa
Jika engkau ingin menghancurkan alam kutuklah seks, kutuklah cinta, ciptakan konsep moral di sekitar kehidupan. Konsep moral itu, bagaimanapun indahnya penampilan mereka, akan seperti kuburan marmer dan engkau akan berada di sana di dalamnya. Beberapa pemabuk mungkin berpikir bahwa engkau tahu apa itu hidup, bahwa engkau tahu bagaimana hidup, tapi siapa pun yang berada dalam keadaan sadarNYA berada dalam keadaan tiada-diri bahkan untuk memanggilmu hidup. Moralitasmu adalah semacam kematian: sebelum kematian membunuhmu, masyarakat membunuhmu.

\english
If you want to destroy nature condemn sex, condemn love, create moral concepts around life. Those moral concepts, howsoever beautiful they look, will be like marble graves and you will be there inside them. Some-drunkard may think that you know what life is, that you know how to live, but anyone who is in HIS state of awareness is in no state even to call you alive. Your morality is a sort of death: before death kills you, the society kills you.

\bahasa
Itulah sebabnya pesan Chuang Tzu adalah salah satu yang paling berbahaya, paling revolusioner, paling memberontak - karena dia berkata: Hargailah alam! Dan jangan memberikan tujuan apapun kepada alam. Siapakah engkau untuk menciptakan arah dan tujuan? Engkau hanyalah bagian kecil, sel atom. Siapakah engkau untuk memaksa keseluruhan untuk bergerak sesuai dengan kehendakmu?

\english
That is why Chuang Tzu's message is one of the most dangerous, the most revolutionary, the most rebellious - because he says: Allow nature! And don't give any goal to nature. Who are you to create goals and purposes? You are just a tiny part, an atomic cell. Who are you to force the whole to move according to you?

\bahasa
Hal ini adalah yang paling berbahaya bagi orang yang religius, untuk orang yang bermoral puritan. Ini adalah pesan yang paling berbahaya. Ini berarti menghancurkan semua rintangan, membiarkan alam meletus - itu berbahaya.

\english
This is most dangerous for people who are religious, for people who are moralistic puritans. This is a most dangerous message. This means break all the barriers, allow nature to erupt - dangerous it is.

\bahasa
Suatu ketika seorang kepala perawat sedang mengenalkan seorang perawat baru, yang baru saja datang dari perguruan tinggi, ke rumah sakit. Dia membawanya ke rumah sakit untuk menunjukkannya rumah sakit itu kepadanya. Dia mengenalkan berbagai bangsal: ini adalah bangsal kanker, ini adalah bangsal tuberkulosis, dan lainnya dan lainnya. Lalu dia sampai di sebuah aula besar, dan dia berkata: Lihat, dan ingatlah baik-baik, ini adalah bangsal yang paling berbahaya, inilah bangsal yang berbahaya. Perawat baru itu melihat, tapi dia tidak dapat melihat apa bahayanya. Jadi dia bertanya: Ada apa? Mengapa bangsal ini paling berbahaya? Bahkan di bangsal kanker engkau tidak mengatakan bahwa itu berbahaya. Kepala perawat tertawa dan dia berkata: Orang-orang ini hampir sembuh, karena itulah inilah bangsal paling berbahaya. Jadi waspadalah - kesehatan selalu berbahaya.

\english
I have heard that a head nurse was introducing a new nurse, who had just come from college, to the hospital. She was taking her around the hospital to show it to her. She introduced the various wards: this is a cancer ward, this is a tuberculosis ward, and others and others. Then she came to a big hall, and she said: Look, and remember well, this is the most dangerous ward of all, this is the dangerous ward. The new nurse looked, but she couldn't see what the danger was. So she asked: What is the matter? Why is this the most dangerous ward? Even in the cancer ward you didn't say that it was dangerous. The head nurse laughed and she said: These people are almost healthy, that's why this is the most dangerous ward. So be alert - health is always dangerous.

\bahasa
Para imam agama takut dengan kesehatan karena kesehatan tidak bermoral di mata mereka. Engkau mungkin pernah mendengar atau mungkin belum pernah mendengar tentang salah satu pemikir abad ini, seorang pemikir Jerman, yang sangat terkenal pada zamannya - Count Keyserling. Dia dianggap filsuf religius yang hebat dan dia menulis dalam buku hariannya: Kesehatan adalah hal yang paling tidak bermoral. Karena kesehatan adalah energi, dan energi itu menyenangkan, energi adalah kenikmatan, energi adalah cinta, energi adalah seks, energi adalah segalanya yang alami. Hancurkan energi, buatlah itu lemah dan redup. Oleh karena itu begitu banyak puasa - hanya untuk menghancurkan energi, hanya untuk mencegah begitu banyak energi dari kebangkitan sehingga itu mulai meluap.

\english
Priests are afraid of health because health is immoral in their eyes. You may or may not have heard of one of the thinkers of this century, a German thinker, very famous in his day - Count Keyserling. He was thought to be a great religious philosopher and he wrote in his diary: Health is the most immoral thing. Because health is energy, and energy is delight, energy is enjoyment, energy is love, energy is sex, energy is everything that is natural. Destroy the energy, make it feeble and dim. Hence so many fasts - just to destroy the energy, just to prevent so much energy from arising that it starts overflowing.

\bahasa
Orang beragama selalu menganggap kesehatan itu berbahaya. Kemudian menjadi tidak sehat menjadi tujuan spiritual.

\english
Religious people have always thought that health is dangerous. Then to be unhealthy becomes a spiritual goal.

\bahasa
Aku ulangi lagi, Chuang Tzu sangat memberontak. Dia berkata: Alam, energi dan ekstase yang datang meluap, dan keseimbangan yang terjadi secara spontan, sudah cukup. Tidak perlu usaha.

\english
I repeat again, Chuang Tzu is very rebellious. He says: Nature, energy and the ecstasy that comes by overflowing, and the balance that happens spontaneously, is enough. There is no need for effort.

\bahasa
Begitu banyak keindahan terjadi di alam tanpa usaha apapun: mawar itu indah tanpa usaha apapun, seekor burung elang malam terus bernyanyi tanpa usaha apapun .... Lihatlah seekor rusa, hidup, penuh energi, cepat. Lihatlah seekor kelinci, sangat waspada, sangat sadar, sehiingga bahkan seorang Buddha pun dapat menjadi cemburu.

\english
So much beauty happens all around in nature without any effort: a rose is beautiful without any effort, a cuckoo goes on singing without any effort.... Look at a deer, alive, full of energy, fast. Look at a hare, so alert, so aware, that even a Buddha may become jealous.

\bahasa
Lihatlah alam semuanya begitu sempurna. Bisakah engkau memperbaiki sekuntum mawar? Dapatkah engkau memperbaiki alam dengan cara apa pun? Hanya manusia yang salah di suatu tempat. Jika mawar itu indah tanpa usaha apapun, mengapa tidak manusia? Apa yang salah dengan manusia? Jika bintang tetap indah tanpa usaha apapun, tanpa sutra yoga Patanjali, mengapa tidak manusia? Manusia adalah bagian dari alam, sama seperti bintang-bintang. Jadi Chuang Tzu berkata: Jadilah alami, dan engkau akan berbunga. Jika pemahaman ini memasuki dirimu, lebih dalam dan lebih dalam dan lebih dalam, maka semua usaha menjadi tidak berarti. Maka engkau tidak terus-menerus membuat pengaturan untuk masa depan, maka engka tinggal di sini dan sekarang, maka momen ini adalah segalanya, maka momen ini adalah kekekalan. Dan Kebuddhaan sudah terjadi, engkau sudah menjadi Buddha. Satu-satunya yang hilang adalah engkau belum memberinya kesempatan untuk berbunga karena engkau terlibat dalam proyekmu sendiri.

\english
Look at nature everything is so perfect. Can you improve upon a rose? Can you improve on nature in any way? Only man has gone wrong somewhere. If the rose is beautiful without any effort on its part, why not man? What is wrong with man? If stars remain beautiful without any effort, without any of Patanjali's yoga sutras, why not man? Man is part of nature, just as stars are. So Chuang Tzu says: Be natural, and you will flower. If this understanding enters you, deeper and deeper and deeper, then all effort becomes meaningless. Then you are not constantly making arrangements for the future, then you live here and now, then this moment is all, then this moment is eternity. And Buddhahood is already the case, you are already a Buddha. The only thing that is missing is that you have not given it any chance to flower because you are so engaged in your own projects.

\bahasa
Bunga mekar tanpa usaha karena energi tidak hilang dalam proyek apapun; Bunga tidak merencanakan masa depan, bunga itu ada disini dan sekarang. Jadilah seperti bunga, jadilah seperti burung, jadilah seperti pohon, sungai, atau samudera - tapi jangan seperti manusia. Karena manusia telah salah pada suatu tempat.

\english
A flower flowers without any effort because the energy is not dissipated in any projects; the flower is not planning for the future, the flower is here and now. Be like a flower, be like a bird, be like a tree, a river, or the ocean - but don't be like a man. Because man has gone wrong somewhere.

\bahasa
Alam dan alami - tanpa usaha alami, alami secara spontan - itulah inti dari semua pengajaran yang akan diberikan Chuang Tzu kepadamu.

\english
Nature and to be natural - effortlessly natural, spontaneously natural - that is the essence of all the teaching that Chuang Tzu is going to give to you.

\bahasa
Sekarang kita akan masuk sutra-nya. Dengarkan setiap kata sedalam mungkin, karena pikiranmu akan menciptakan rintangan, pikiranmu tidak akan membiarkanmu mendengarkan. Pikiran adalah masyarakat di dalam dirimu. Masyarakat sangat licik: tidak hanya di luar dirimu, itu telah merasuk di dalam dirimu. Itulah pikiranmu, dan itulah sebabnya semua orang yang tahu bertentangan dengan pikiran dan mendukung alam. Karena pikiran adalah hal yang buatan, ditanamkan dalam dirimu oleh masyarakat. Jadi saat engkau mendengarkan Chuang Tzu, pikiranmu akan menciptakan hambatan. Pikiranmu tidak ingin mendengarkan karena apa yang dia katakan sangat bertentangan dengan pikiran. Jika engkau mengizinkannya, jika engkau menyisihkan pikiranmu dan mengizinkannya menembus dirimu, mendengarkan akan menjadi meditasi, mendengarkan akan mengubah dirimu. Tidak ada hal lain yang harus dilakukan, hanya mendengarkan.

\english
Now we will enter his sutra. Listen to every word as deeply as possible, because your mind will create barriers, your mind will not allow you to listen. The mind is society within you. Society is very cunning: it is not only outside you, it has penetrated within you. That is what your mind is, and that is why all those who know are against the mind and for nature. Because mind is an artificial thing, implanted in you by society. So when you listen to Chuang Tzu your mind will create barriers. Your mind will not like to listen because what he says is so against the mind. If you allow it, if you put aside your mind and allow it to penetrate you, the very listening will become meditation, the very listening will transform you. There is no other thing to be done, just listening.

\bahasa
Chuang Tzu percaya pada pemahaman, bukan dalam meditasi. Dan jika aku mengatakan bahwa engkau harus bermeditasi, itu hanya karena aku merasa bahwa pemahaman itu sangat sulit untukmmu. Meditasi tidak akan membawamu ke tujuan - tidak ada metode yang dapat membawammu ke tujuan. Tidak ada metode, tidak ada teknik. Meditasi hanya akan membantumu untuk memahami. Itu tidak akan membawamu ke kebenaran, itu hanya akan menghancurkan pikiran, sehingga setiap kali ada kebenaran, engkau dapat melihatnya.

\english
Chuang Tzu believes in understanding, not in meditation. And if I say you have to meditate, it is only because I feel that understanding is so difficult for you. Meditation will not lead you to the goal - no method can lead you to the goal. There exists no method, no technique. Meditation will only help you to understand. It will not lead you to the truth, it will only destroy the mind, so that whenever there is truth, you can see it.

\bahasa
CHU'I SI JURU GAMBAR DAPAT MENGGAMBAR LINGKARAN LEBIH SEMPURNA DENGAN TANGAN HAMPA DARIPADA DENGAN SEBUAH KOMPAS.

\english
CHU'I THE DRAFTSMAN COULD DRAW MORE PERFECT CIRCLES FREEHAND THAN WITH A COMPASS.

\bahasa
Chuang Tzu berbicara tentang seorang juru gambar yang bernama Chu'i; yang dapat menggambar lingkaran lebih sempurna secara bebas daripada dengan kompas. Sungguh, kompas sangat dibutuhkan karena engkau takut. Jika engkau tidak engkau sendiri dapat menggambar lingkaran sempurna tanpa bantuan apapun.

\english
Chuang Tzu talks about a draftsman of the name Chu;i, who could draw more perfect circles freehand than with a compass. Really, the compass is needed because you are afraid. If you are not afraid you yourself can draw a perfect circle without any help.

\bahasa
Di alam lingkaran ada dimana-mana, semuanya bergerak di jalur melingkar. Lingkaran adalah fenomena termudah di alam - dan tidak ada kompas yang digunakan. Bintang-bintang tidak berkonsultasi dengan peta, mereka tidak membawa kompas dan mereka terus bergerak dalam lingkaran. Jika engkau memberi mereka kompas dan peta, aku yakin mereka akan tersesat - mereka tidak akan tahu harus pergi ke mana dan apa yang harus dilakukan.

\english
In nature circles exist everywhere, everything moves on a circular path. The circle is the easiest phenomenon in nature - and no compass is used. The stars don't consult a map, they don't carry a compass and they go on moving in a circle. If you give them compasses and maps, I am certain they will be lost - they will not know where to go and what to do.

\bahasa
Engkau pasti pernah mendengar cerita tentang kelabang? Seekor kelabang berjalan dengan seratus kaki. Seorang katak, seorang filsuf, melihat kelabang itu, dia melihat dan mengamati dan dia menjadi sangat terganggu; Sangat sulit berjalan bahkan dengan empat kaki, tapi kelabang ini berjalan dengan seratus kaki. Ini adalah sebuah keajaiban! Bagaimana sang kelabang menentukan kaki mana yang harus melangkah terlebih dahulu, lalu mana yang berikutnya dan kemudian kaki yang mana setelah itu? Dan seratus kaki! Jadi katak menghentikan kelabang dan mengajukan pertanyaan: Aku adalah seorang filsuf dan aku bingung dengan dirimu. Sebuah masalah telah muncul yang tidak dapat aku selesaikan. Bagaimana engkau berjalan Bagaimana engkau mengendalikannya? Sepertinya tidak mungkin! Sang kelabang berkata: Aku telah berjalan sepanjang hidupku, tapi aku belum pernah memikirkannya. Sekarang setelah engkau bertanya, aku akan memikirkannya dan kemudian aku akan memberi tahumu.

\english
You must have heard the story of the centipede? A centipede walks with a hundred legs. A frog, a philosopher, saw the centipede, he looked and watched and he became very troubled; it is so difficult to walk even with four legs, but this centipede was walking with one hundred legs. This was a miracle! How did the centipede decide which leg to move first, and then which one next and then which one after that? And one hundred legs! So the frog stopped the centipede and asked a question: I am a philosopher and I am puzzled by you. A problem has arisen which I cannot solve. How do you walk? How do you manage it at all? It seems impossible! The centipede said: I have been walking all my life, but I have not thought about it. Now that you ask, I will think about it and then I will tell you.

\bahasa
Untuk pertama kalinya pikiran memasuki kesadaran sang kelabang. Sungguh, katak itu benar - kaki mana yang harus bergerak terlebih dulu? Kelabang itu berdiri di sana selama beberapa menit, tidak dapat bergerak, bergoyang-goyang, dan terjatuh. Dan dia berkata kepada si katak: Tolong jangan tanya kelabang lain pertanyaan ini. Aku telah berjalan sepanjang hidupku dan itu tidak pernah menjadi masalah, dan sekarang engkau telah membunuhku sepenuhnya! Aku tidak dapat bergerak. Dan seratus kaki untuk bergerak! Bagaimana aku dapat mengaturnya?

\english
For the first time thought entered the centipede's consciousness. Really, the frog was right - which leg should be moved first? The centipede stood there for a few minutes, couldn’t move, wobbled, and fell down. And he said to the frog: Please don't ask another centipede this question. I have been walking throughout my life and it was never a problem, and now you have killed me completely! I cannot move. And a hundred legs to move! How can I manage?

\bahasa
Hidup bergerak dalam lingkaran sempurna, hidup bergerak dengan sempurna, tidak ada masalah. Chuang Tzu mengatakan tentang Chu'i bahwa ia dapt menggambar lingkaran yang lebih sempurna secara bebas daripada dengan kompas. Engkau membutuhkan kompas karena engkau tidak yakin akan kehidupan; engkau membutuhkan moralitas, pedoman, prinsip, Alkitab, Alquran, Gitas untuk mengarahkanmu karena engkau tidak yakin akan kekuatan batin. Itulah hidupmu. Dan Alkitab, Alquran, dan Gitas ini, mereka telah menciptakan situasiuntukmu sama bagaimana si katak menciptakannya untuk kelabang.

\english
Life moves in a perfect circle, life moves perfectly, there is no problem. Chuang Tzu says of Chu'i that he could draw more perfect circles freehand than with a compass. You need a compass because you are not confident of life; you need moralities, precepts, principles, Bibles, Korans, Gitas to direct you because you are not confident of the inner force. That is your life. And these Bibles, Korans, and Gitas, they have created the same situation for you that the frog created for the centipede.

\bahasa
Begitu banyak pedoman yang harus diikuti, begitu banyak prinsip yang harus dikelola - begitu banyak konsep moral. Engkau memiliki banyak hal yang dibebankan kepadamu sehingga kehidupan batinmu tidak dapat spontan. Engkau tersesat, bukan karena kekuatan jahat, tapi karena para humanis. Bukan Iblis yang menuntunmu ke arah yang salah, itu adalah para imam agamamu, pemimpinmu, orang-orang kudusmu.

\english
So many precepts to be followed, so many principles to be managed - so many moral concepts. You have so many things imposed on you that your inner life cannot be spontaneous. You go astray, not because of any evil force, but because of the do-gooders. It is not a Devil which is leading you towards wrong, it is your priests, your leaders, your so-called saints.

\bahasa
Ini sangat sulit. Mudah untuk percaya pada Iblis, jadi engkau membuang semua tanggung jawab kepada Iblis. Tidak ada Iblis, aku katakan. Dan itulah yang dikatakan juga oleh Chuang Tzu.

\english
This is very difficult. It is easy to believe in a Devil, so you throw all the responsibility onto the Devil. There is no Devil, I tell you. And that is what Chuang Tzu is saying also.

\bahasa
Chuang Tzu berkata: Tidak ada Tuhan, tidak ada Iblis, hanya kehidupan yang ada. Para imam menciptakan Tuhan dan para imam menciptakan Iblis karena para imam menciptakan perbedaan antara yang salah dan benar. Dan begitu perbedaan itu masuk ke dalam pikiranmu, engkau tidak akan pernah benar. Alam benar. Setelah perbedaan masuk ke dalam pikiranmu bahwa ini salah dan itu benar, engkau tidak akan pernah benar, engkau tidak akan pernah merasa nyaman, engkau tidak akan pernah merasa rileks, engkau akan selalu merasa tegang. Dan apa pun yang engkau lakukan akan salah karena perbedaan itu menciptakan kebingungan. Seluruh hidup begitu hening dan meditatif, mengapa begitu banyak usaha yang dibutuhkan untukmu? Itu karena ada perbedaan.

\english
Chuang Tzu says: There is no God, there is no Devil, only life exists. Priests create God and priests create the Devil because priests create the distinction between wrong and right. And once the distinction enters your mind you will never be right. Nature is right. Once the distinction enters your mind that this is wrong and that is right, you will never be right, you will never be at ease, you will never feel relaxed, you will always be tense. And whatsoever you do will be wrong because the distinction creates confusion. The whole of life is so silent and meditative, why is so much effort needed for you? It is because there is distinction.

\bahasa
CHU'I SI JURU GAMBAR DAPAT MENGGAMBAR LINGKARAN LEBIH SEMPURNA DENGAN TANGAN HAMPA DARIPADA DENGAN SEBUAH KOMPAS. Jika engkau tidak sadar diri, hidupmu bergerak secara otomatis. Kompas itu sadar diri: engkau melakukan sesuatu dengan sadar diri dan engkau akan berada dalam masalah. Engkau berbicara, sepanjang hari engkau terus mengobrol dengan teman, dan tidak ada masalah. Tapi jika aku memintamu datang kemari dan berbicara dari kursi ini ke teman-teman yang telah berkumpul di sini, engkau akan berada di posisi yang sama seperti kelabang. Namun engkau telah berbicara seumur hidupmu dan tidak pernah ada masalah.

\english
CHU'I THE DRAFTSMAN COULD DRAW MORE PERFECT CIRCLES FREEHAND THAN WITH A COMPASS. If you are not self-conscious your life moves automatically. That compass is self consciousness: you do anything self-consciously and you will be in trouble. You talk, the whole day you go on chattering with friends, and there is no problem. But if I ask you to come here and talk from this chair to the friends who have gathered here, you will be in the same position as the centipede. And yet you have been talking your whole life and there was never a problem.

\bahasa
Mengapa masalah ini masuk? Masalahnya masuk karena sekarang engkau sadar diri. Sekarang begitu banyak orang melihatmu, memperhatikanmu, bahwa sekarang engkau tidak dapat merasa nyaman dan spontan. Sekarang engkau memproyeksikan, sekarang engkau ingin merencanakan, sekarang engkau ingin orang-orang menyukaimu. Apa pun yang engkau katakan, engkau ingin mereka terkesan - sekarang engkau sadar diri.

\english
Why does this problem come in? The problem comes in because now you are self-conscious. Now so many persons are looking at you, watching you, that now you cannot be at ease and spontaneous. Now you project, now you want to plan, now you want the people to like you. Whatsoever you say, you would like them to be impressed - now you are self conscious.

\bahasa
Jika tidak semua orang adalah pembicara, terlahir sebagai seorang pembicara. Orang terus berbicara dan tidak pernah ada masalah. Tapi begitu engkau menempatkan mereka di mimbar dan memberitahu mereka untuk berbicara dengan penonton, ada yang tidak beres. Apa yang salah? Tidak ada yang berubah tapi sadardiri telah masuk, dan sadar diri adalah masalahnya.

\english
Otherwise everybody is a talker, a born talker. People go on talking and there is never any problem. But once you put them in a pulpit and tell them to talk to an audience, something goes wrong. What goes wrong? Nothing has changed but self-consciousness has entered, and self consciousness is the problem.

\bahasa
JEMARINYA MELAHIRKAN BENTUK SPONTAN ENTAH DARI MANA. PIKIRANNYA LIAR SELAGI BEBAS DAN TANPA MEMPEDULIKAN APA YANG SEDANG DILAKUKANNYA.

TIDAK ADA LATIHAN YANG DIBUTUHKAN, PIKIRANNYA SANGAT SEDERHANA DAN TIDAK MENGENAL HAMBATAN.

\english
HIS FINGERS BROUGHT FORTH SPONTANEOUS FORMS FROM NOWHERE. HIS MIND WAS MEANWHILE FREE AND WITHOUT CONCERN WITH WHAT HE WAS DOING.

NO APPLICATION WAS NEEDED, HIS MIND WAS PERFECTLY SIMPLE AND KNEW NO OBSTACLE.

\bahasa
JEMARINYA MELAHIRKAN BENTUK SPONTAN ENTAH DARI MANA. Entah dari mana berarti dimana-mana, entah dari mana berarti kekosongan yang terinti; entah dari mana berarti sumber yang terinti, dasar kehidupan.

\english
HIS FINGERS BROUGHT FORTH SPONTANEOUS FORMS FROM NOWHERE. Nowhere means everywhere, nowhere means the ultimate void; nowhere means the ultimate source, the very basis of life.

\bahasa
Dari mana engkau bernafas begitu sempurna? Chuang Tzu mengatakan bahwa engkau tidak bernafas, sebaliknya,'Itu' bernafas dirimu. ENGKAU tidak bernafas, karena apa yang harus engkau lakukan dengannya? Tidak ada satupun. "Aku bernafas' adalah gagasan yang salah, lebih baik mengatakan: Alam,' itu ', bernafas diriku. Kemudian seluruh pola berubah, maka keseluruhan penekanannya adalah pada alam, bukan pada engkau, bukan pada ego, tapi pada 'Itu', yang maha luas, yang tak terbatas yang mengelilingimu, yang menjadi dasar, yang begitu menjadi dasar - 'itu' bernafas dirimu.

\english
From where are you breathing so perfectly? Chuang Tzu says you are not breathing, rather,'It' breathes you. YOU are not breathing, because what do you have to do with it? Nothing.'I am breathing' is a false notion, it would be better to say: Nature, 'It', breathes me. Then the whole gestalt changes, then the whole emphasis is on nature, not on you, not on the ego, but on 'It', the vast, the infinite that surrounds you, the basis, the very basis - 'It' breathes you.

\bahasa
Ketika engkau jatuh cinta, apakah itu benar-benar dirimu yang jatuh cinta - atau apakah 'Itu' jatuh cinta melalui dirimu? Saat engkau marah, apakah engkau marah? Karena bila ada kemarahan, engkau tidak lagi ada; ketika ada cinta, engkau tidak lagi ada. Dalam kemarahan, cinta, dalam emosi yang penuh gairah, engkau tidak lagi ada. Dalam apapun yang hidup, engkau menghilang; maka 'Itu' ada - Tao.

\english
When you fall in love, is it really you who falls in love - or does 'It' fall in love through you? When you are angry, are you angry? Because when there is anger, you are not; when there is love, you are not. In anger, in love, in any passionate emotion, you are not. In anything alive, you disappear; then 'It' exists - Tao.

\bahasa
Jadi seorang Tao adalah orang yang mengerti bahwa 'aku' adalah hal yang paling tidak berguna. Ini hanya menimbulkan masalah dan tidak ada yang lain - jadi dia menjatuhkannya. Sungguh tidak perlu menjatuhkannya, begitu dia mengerti, itu tetes - tidak ada 'aku'. Lalu dia hidup, lalu dia makan, lalu dia cintai, lalu dia tidur, tapi tidak ada 'aku'. 'Itu' hidup melalui dia. Maka tidak ada beban dan tidak ada ketegangan dan tidak ada kecemasan; maka dia menjadi anak kecil; maka pikirannya bebas, tanpa kepedulian.

\english
So a man of Tao is one who has come to understand that 'I' is the most useless thing. It only creates problems and nothing else - so he drops it. Really there is no need to drop it, once he understands, it drops - there is no 'I'. Then he lives, then he eats, then he loves, then he sleeps, but there is no 'I'.'It' lives through him. Then there is no burden and no tension and no anxiety; then he becomes a child; then his mind is free, without concern.

\bahasa
Engkau tidak dapat melakukan apapun tanpa kepedulian. Apa pun yang engkau lakukan, ego masuk, perhatian masuk, dan kemudian ada kegelisahan.

\english
You cannot do anything without concern. Whatsoever you do, the ego comes in, concern comes in, and then there is anxiety.

\bahasa
Lihatlah fenomena ini: ahli bedah melakukan operasi, dan dia adalah ahli bedah yang sempurna. Tapi jika istrinya berada di meja operasi, dia tidak dapat melakukan operasi; tangannya gemetar. Di lain waktu, dia bekerja seperti mekanisme yang sempurna, tapi saat istrinya berada di atas meja, dia tidak dapat melakukan operasi - beberapa ahli bedah lain dibutuhkan.

\english
Look at this phenomenon: a surgeon operates, and he is a perfect surgeon. But if his wife is on the operation table, he cannot operate; his hand trembles. At other times, he works like a perfect mechanism, but when his wife is on the table, he cannot operate - some other surgeon is needed.

\bahasa
Apa yang telah terjadi? Kepedulian sudah masuk. Dengan pasien lain tidak ada keprihatian, dia terlepas. Dia tidak peduli dengan cara ini atau itu, dia hanyalah seorang ahli bedah, sebuah kekuatan alami, bekerja. Pikiran tidak ada di sana; dia sempurna. Tapi sekarang istrinya ada di sana, kepedulian sudah masuk: Akankah operasi itu berhasil atau tidak? Apakah aku dapat menyelamatkan istriku atau tidak? Sekarang, masalah-masalah ini ada di sana, pikirannya memiliki sebuah kepedulian - lalu tangannya gemetar.

\english
What has happened? Concern has entered. With other patients there was no concern, he was detached. He was not concerned this way or that, he was simply a surgeon, a natural force, working. The mind was not there; he was perfect. But now that his wife is there, concern has entered: Will the operation succeed or not? Will I be able to save my wife or not? Now, these problems are there, his mind has a concern - then his hand trembles.

\bahasa
Seluruh hidupmu adalah sebuah gemetaran karena engkau telah membawa begitu banyak kepedulian; dan kemudian engkau tidak dapat menggambar lingkaran yang sempurna.

\english
Your whole life is a trembling because you have been carrying so many concerns; and then you cannot draw a perfect circle.

\bahasa
Dan tulisanmu....

\english
And your writing....

\bahasa
Ada sebuah ilmu tentang membaca tulisanmu dan melaluinya, pikiranmu. Ada dasar yang pasti karena saat engkau menulis, gemetaranmu masuk ke dalamnya. Dan saat engkau menandatangani namamu, engkau yang paling risau. Lalu gemetaranmu ada di sana, dan dengan kaca pembesar gemetar itu dapat diamati, dapat dideteksi. Gemetar itu dapat menunjukkan banyak hal tentang dirimu karena apa pun yang engkau lakukan, ENGKAU melakukannya. Itu akan membawa ENGKAU, tui akan membawa indikasi tentang ENGKAU. Hanya dengan melihat tulisan tanganmu banyak yang dapat diketahui tentang kepribadianmu.

\english
There is a science about reading your writing and through it, your mind. There is a definite basis to it because when you write, your trembling enters it. And when you sign your name you are the most concerned. Then your trembling is there, and with a magnifying glass that trembling can be observed, can be detected. That trembling can show much about you because whatsoever you are doing, YOU are doing it. It will carry YOU, it will carry indications about YOU. Just by seeing your handwriting much can be known about your personality.

\bahasa
Jika seorang Buddha memberi tanda tangan, itu akan menjadi sangat berbeda. Tidak akan ada gemetar karena tidak ada kepedulian. Dan bahkan melalui tanda tangan itu dapat dikatakan apakah itu milik seorang Buddha atau tidak.

\english
If a Buddha signs, it is going to be absolutely different. There will be no trembling because there is no concern. And even through the signature it can be said whether it belongs to a Buddha or not.

\bahasa
Apa pun yang engkau lakukan, gemetaranmu mengikuti dirimu seperti bayangan. Siapa yang menciptakan gemetaran ini?

\english
Whatsoever you do, your trembling follows you like a shadow. Who is creating this trembling?

\bahasa
Engkau datang kepadaku dan engkau berkata: Aku tidak damai; Pikiranku tidak hening. Bagaimana itu bisa hening kecuali engkau menjatuhkan kepedulianmu? Engkau ingin pikiranmu diam, engkau ingin pikiranmu dibuat hening, jelas, transparan. Tanpa menjatuhkan perhatian itu tidak mungkin karena masih akan ada sebuah gemetaran.

\english
You come to me and you say: I am not at peace; my mind is not silent. How can it be unless you drop your concern? You want your mind to be stilled, you want your mind to be made silent, clear, transparent. Without dropping the concern it is impossible because there will still be a trembling.

\bahasa
Satu-satunya hal yang dapat dilakukan tanpa mengubah kepedulianmu adalah menekan semua gemetaran didalam. Jadi jika engkau menyaksikan engkau akan merasa bahwa di permukaan semuanya pasif, tenang, tapi jauh di lubuk hatimu engkau gemetaran, terus gemetar. Jauh didalam ketakutan dan gemetaran terus berlanjut. Mereka lahir karena kepedulian.

\english
The only thing that can be done without changing your concern is to suppress all the trembling inside. So if you watch you will feel that on the surface everything is placid, calm, but deep down you are trembling, continuously trembling. Deep down fear and trembling continue. They are born out of concern.

\bahasa
Dan apa yang menjadi kepeduliannya? Ini tentang bagaimana orang lain terkesan olehmu. Tapi kenapa engkau begitu mengkhawatirkan orang lain? Begitu khawatir sehingga engkau tidak dapat hidup? Semua orang bertanya-tanya apa yang dipikirkan orang lain tentang mereka, dan hal yang sama terjadi pada orang lain. Mereka khawatir tentang engkau, dan engkau khawatir tentang mereka.

\english
And what is the concern? It is about how others are impressed by you. But why are you so worried about others? So worried that you cannot live at all? Everybody is wondering what others are thinking about them, and the same is the case with others. They are worried about you, and you are worried about them.

\bahasa
Suatu hari terjadi Mulla Nasrudin sedang berjalan di jalan setapak. Jalan itu sepi, matahari telah terbenam, dan kegelapan turun. Tiba-tiba dia merasa takut karena beberapa orang datang, dalam sebuah gerombolan, dan dia berpikir: Mereka pasti dacoits, perampok, dan tidak ada orang lain di sini, hanya diriku sendiri. Jadi dia melompati dinding yang berada tepat di dekatnya dan mendapati dirinya berada di sebuah kuburan. Kuburan yang baru digali berada di sana sehingga dia naik ke dalamnya, menenangkan diri, memejamkan mata, dan menunggu orang-orang itu melewatinya agar bisa pulang. Tapi orang-orang itu juga melihat seseorang ada di sana. Mulla tiba-tiba melompat, jadi mereka juga menjadi takut. Apa masalahnya? Apakah seseorang bersembunyi di sana, atau melakukan sesuatu yang jahat? Jadi mereka semua melompati tembok. Sekarang, Mulla yakin: Aku benar, aku yakin benar, mereka adalah orang-orang yang berbahaya. Sekarang tidak ada lagi yang dapat dilakukan; Aku harus berpura-pura mati. Jadi dia berpura-pura mati. Dia menghentikan napasnya karena engau tidak dapat merampok atau membunuh orang mati. Tapi orang-orang itu melihat orang itu melompat sehingga mereka menjadi sangat khawatir. Apa yang dia lakukan. Mereka berkumpul di sekitar, melihat ke dalam kuburan, dan mereka berkata: Apa maksudnya? Apa yang sedang engkau lakukan? Mengapa engkau di sini? Mulla membuka matanya, menatap mereka, lalu menjadi yakin bahwa tidak ada bahaya. Dia tertawa, dan berkata: Sekarang, inilah masalahnya, masalah filosofis. Kalian bertanya kepadaku mengapa aku berada di sini, dan aku juga ingin bertanya mengapa kalian ada di sini? Aku di sini karena kalian, dan kalian disini karena aku!

\english
Once it happened that Mulla Nasrudin was walking on a path. It was a lonely path, the sun had set, and darkness was descending. Suddenly he felt afraid because a few people were coming, in a band, and he thought: These must be dacoits, robbers, and there is nobody else here, only myself. So he jumped over a wall that was just nearby and found himself in a graveyard. A newly dug grave was there so he climbed into it, somehow calmed himself, closed his eyes, and waited for the people to pass so he could go home. But the people had also seen that somebody was there. Mulla had jumped suddenly, so they also became afraid. What was the matter? Was somebody hiding there, or doing something mischievous? So they all jumped over the wall. Now, Mulla was certain: I was right, I inferred right, they are dangerous people. Now nothing more can be done; I must pretend that I am dead. So he pretended. He stopped his breathing because you cannot rob or kill a dead man. But the people had seen the man jump so they became very worried. What was he doing. They gathered around, looked in the grave, and they said: What is the idea? What are you doing? Why are you here? Mulla opened his eyes, looked at them, then became certain that there was no danger. He laughed, and said: Now, here is a problem, a very philosophical problem. You ask me why I am here, and I would like to ask why you are here. I am here because of you, and you are here because of me!

\bahasa
Itu adalah lingkaran setan: engkau takut pada orang lain, orang lain takut padamu, dan seluruh hidupmu menjadi berantakan. Keluar dari omong kosong ini, keluar dari lingkaran setan ini, jangan perduli dengan orang lain. Hidupmu sudah cukup, jangan perduli dengan orang lain. Dan aku katakan bahwa jika engkau dapat hidup tidak peduli, hidupmu akan berbunga, dan kemudian orang lain dapat berbagi di dalamnya. Engkau ingin berbagi, dan engkau dapat memberi banyak hal kepada orang lain, tapi pertama-tama engkau harus berhenti memikirkan orang lain dan apa yang mereka pikirkan tentang dirimu.

\english
It is a vicious circle: you are afraid of others, others are afraid of you, and your whole life becomes a mess. Drop out of this nonsense, drop out of this vicious circle, don't be concerned with others. Your life is enough, don't be concerned with others. And I tell you that if you can live unconcerned your life will flower, and then others can share in it. You would like to share, and you can give much to others, but first you must stop thinking about others and what they are thinking about you.

\bahasa
'tentang' ini sangat berbahaya. Tidak ada yang merasa nyaman, tidak ada yang merasa di rumah. Karena orang lain, semua orang mengejar orang lain - dan hidup telah menjadi neraka.

\english
This 'about' is very dangerous. Nobody is at ease, nobody is at home. Because of others, everybody is chasing everybody else - and life has become a hell.

\bahasa
JEMARINYA MELAHIRKAN BENTUK SPONTAN ENTAH DARI MANA. Pikiranya SEMENTARA ITU BEBAS DAN TANPA MEMPEDULIKAN APA YANG SEDANG DILAKUKANNYA.

\english
HIS FINGERS BROUGHT FORTH SPONTANEOUS FORMS FROM NOWHERE. His mind WAS MEANWHILE FREE AND WITHOUT CONCERN WITH WHAT HE WAS DOING.

\bahasa
LAKUKAN! Jangan khawatir dengan apa yang sedang engkau lakukan - lakukan dengan sepenuh hati bahwa perbuatan itu menjadi kebahagiaan sejati. Dan jangan memikirkan hal-hal besar, tidak ada yang namanya besar atau kecil. Jangan berpikir bahwa engkau akan melakukan perbuatan-berbuatan besar, bermain musik besar, melukis lukisan-lukisan besar, bahwa engkau akan menjadi seorang Picasso atau Van Gogh, atau sesuatu yang lain - seorang penulis besar, seorang Shakespeare, atau Milton. Tidak ada apa-apa - tidak ada hal besar, tidak ada hal kecil. Ada manusia besar dan manusia kecil tapi hal-hal tidak besar dan kecil. Dan seorang manusia besar adalah orang yang membawa kebesarannya pada setiap hal kecil yang dia lakukan: dia makan dengan cara yang besar, dia berjalan dengan cara yang besar, dia tidur dengan cara yang besar. Dia membawa kualitas kebesaran untuk segalanya.

\english
DO! Don't be concerned about what you are doing - do it so wholeheartedly that the very doing becomes a bliss. And don't think of great things, there is no such thing as great or small. Don't think that you are to do great things, play great music, paint great paintings, that you are to become a Picasso or a Van Gogh, or something else - a great writer, a Shakespeare, or a Milton. There is nothing - no great things, no small things. There are great men and small men but things are not great and small. And a great man is one who brings his greatness to every small thing that he is doing: he eats in a great way, he walks in a great way, he sleeps in a great way. He brings the quality of greatness to everything.

\bahasa
Dan apakah kebesaran itu? Alam .... Tidak ada yang lebih besar dari alam.

\english
And what is greatness? Nature.... Nothing is greater than nature.

\bahasa
Makan seperti seorang kaisar. Itu tidak tergantung pada kualitas makanannya, itu tergantung pemakannya, cara dia merayakannya. Bahkan dengan hanya roti, mentega, dan garam, engkau dapat menjadi seorang kaisar.

\english
Eat like an emperor. That doesn't depend upon the quality of the food, it depends on the eater, the way he celebrates it. Even with just bread, butter, and salt, you can be an emperor.

\bahasa
Itu terjadi bahwa Epicurus memiliki sebuah taman, dekat dengan Athena....

\english
It happened that Epicurus had a garden, just near Athens....

\bahasa
Dia juga salah satu manusia paling langka, seperti Chuang Tzu. Dia tidak percaya pada Tuhan, dia tidak percaya pada apapun, karena kepercayaan itu omong kosong. Hanya orang bodoh yang percaya. Orang yang memiliki pemahaman memiliki iman, bukan kepercayaan. Iman berbeda. Iman berarti mempercayai kehidupan, mempercayainya begitu mutlak sehingga seseorang siap untuk pergi bersamanya, di mana saja.

\english
He was also one of the rarest of men, just like Chuang Tzu. He didn't believe in God, he didn't believe in anything, because belief is nonsense. Only foolish people believe. A man of understanding has faith, not belief. Faith is different. Faith means trusting life, trusting it so absolutely that one is ready to go with it, anywhere.

\bahasa
.... Dia memiliki sebuah taman kecil, dan dia tinggal di sana bersama murid-muridnya. Orang-orang mengira dia ateis, tidak bermoral. Dia tidak percaya kepada Tuhan, dia tidak percaya akan kita suci, dia tidak percaya pada kuil apapun; dia dulu adalah seorang ateis Tapi dia hidup dengan cara yang begitu besar. Hidupnya luar biasa, luar biasa meski dia tidak punya apa-apa, meski mereka sangat miskin. Raja mendengar tentang mereka, dan ingin melihat bagaimana mereka hidup, dan bagaimana mereka dapat bahagia tanpa kepercayaan. Jika engkau tidak dapat bahagia bahkan dengan kepercayaan kepada Tuhan bagaimana orang-orang ini bisa bahagia tanpa Tuhan?

\english
.... He had a small garden, and he lived there with his disciples. People thought that he was an atheist, immoral. He did not believe in God, he did not believe in scriptures, he did not believe in any temple; he was an atheist. But he lived in such a great way. His life was superb, magnificent even though he had nothing, even though they were very poor. The king heard about them, and wanted to see how they lived, and how they could be happy without belief. If you cannot be happy even with
a belief in God how could these people be happy without God?

\bahasa
Jadi dia datang suatu sore untuk mengunjungi tamannya Epicurus.

\english
So he came one evening to visit Epicurus' garden.

\bahasa
Dia benar-benar terkejut, kagum - itu adalah keajaiban. Mereka sama sekali tidak punya apa-apa, hampir tidak punya sama sekali, tapi mereka hidup seperti kaisar. Seperti Dewa mereka hidup. Seluruh hidup mereka adalah sebuah perayaan. Ketika mereka pergi ke sungai untuk mandi, itu bukan sekadar mandi; Itu adalah tarian bersama sungai, semakin selaras dengan sungai. Mereka bernyanyi, dan mereka menari, dan mereka berenang, dan mereka melompat, dan mereka pun menyelam.

\english
He was really surprised, amazed - it was a miracle. They had nothing, almost nothing, but they lived like emperors. Like Gods they lived. Their whole life was a celebration. When they went to the stream to take their bath, it was not simply a bath; it was a dance with the river, it was getting in tune with the river. They sang, and they danced, and they swam, and they jumped, and they dived.

\bahasa
Makan mereka adalah sebuah perayaan, sebuah pesta, dan mereka tidak memiliki apa-apa, hanya roti dan garam, bahkan mentega pun tidak. Tapi mereka merasa sangat bersyukur bahwa hanya untuk menjadi adalah cukup, tidak ada lagi yang dibutuhkan.

\english
Their eating was a celebration, a feast, and they had nothing, just bread and salt, not even butter. But they felt so thankful that just to be was enough, nothing more was needed.

\bahasa
Sang Kaisar sangat terkesan, dan dia bertanya kepada Epicurus: Lain kali aku datang, aku ingin membawa beberapa hadiah untukmu. Apa yang engkau mau? Epicurus berkata: Beri kami waktu untuk berpikir, kami tidak pernah berpikir bahwa seseorang akan memberi kami hadiah, dan kami memiliki begitu banyak karunia dari alam. Tapi jika engkau bersikeras, bawalah sedikit mentega, tidak ada yang lain. Hanya itu sudah cukup.

\english
The emperor was very much impressed, and he asked Epicurus: Next time I come, I would like to bring some gifts for you. What would you like? Epicurus said: Give us time to think, we never thought that anybody would give us gifts, and we have so many gifts from nature. But if you insist, then bring a little butter, nothing else. Just that will do.

\bahasa
Hidup dapat menjadi sebuah perayaan jika engkau tahu bagaimana hidup tanpa kepedulian. Jika tidak, hidup menjadi penyakit dan keburukan yang berkepanjangan yang mencapai puncaknya hanya dalam kematian.

\english
Life can become a celebration if you know how to live without concern. Otherwise life becomes a long prolonged disease and illness which culminates only in death.

\bahasa
.... PIKIRANNYA SEMENTARA ITU BEBAS DAN TANPA MEMPEDULIKAN APA YANG SEDANG DILAKUKANNYA.TIDAK ADA LATIHAN YANG DIBUTUHKAN, PIKIRANNYA SANGAT SEDERHANA DAN TIDAK MENGENAL HAMBATAN.

\english
.... HIS MIND WAS MEANWHILE FREE AND WITHOUT CONCERN, WITH WHAT HE WAS DOING. NO APPLICATION WAS NEEDED – HIS MIND WAS PERFECTLY SIMPLE AND KNEW NO OBSTACLE.

\bahasa
Engkau perlu belajar segalanya karena engkau telah melupakan sifat alamimu sepenuhnya. Sekarang psikolog mengusulkan bahwa harus ada pelatihan untuk cinta, karena orang-orang perlahan-lahan melupakan bagaimana mencintai. Banyak literatur telah muncul: 'Seni dari Cinta', 'Bagaimana mencintai?' Orang-orang sudah benar-benar melupakan orgasme seksual, ekstase seksual. Tidak ada binatang yang membutuhkan latihan apapun! Bahkan pohon pun tampaknya lebih cerdas dari dirimu.


\english
You need to learn everything because you have forgotten your nature completely. Now psychologists are proposing that there must be training for love, because people are by and by forgetting how to love. Much literature has come into existence: 'The Art of Love', 'How to Love?' People have completely forgotten sexual orgasm, sexual ecstasy. No animal needs any training! Even trees seem to be more intelligent than you.

\bahasa
Semuanya harus diajarkan, bahkan dasar-dasar kehidupan pun harus diajarkan. Itu berarti entah bagaimana kita tidak mengakar. Kita telah kehilangan kontak dengan alam, ada jarak.

\english
Everything has to be taught, even the very basics of life have to be taught. That means that somehow we are uprooted. We have lost contact with nature, a gap exists.

\bahasa
Dan jika engkau diajari bagaimana mencintai cintamu itu akan menjadi palsu. Cinta sejati harus spontan. Bagaimana engkau dapat diajarkan untuk mencintai? Jika engkau diajari, maka engkau akan bertindak sesuai aturan dan arus yang alami tidak akan ada di sana.

\english
And if you are taught how to love your love is going to be false. Real love should be spontaneous. How can you be taught to love? If you are taught, then you will act according to the rules and the natural flow will not be there.

\bahasa
Alam tidak mengalir sesuai dengan peraturan-peraturanmu; alam memiliki aturan-aturannya sendiri. Engkau hanya perlu bersamanya dan ia mulai berfungsi. Hari ini tidak terlalu jauh ketika kita harus mengajari orang bagaimana bernafas. Sekarang engkau menertawakannya, tapi jika engkau kembali kebelakang dan bertanya kepada Epicurus, 'Akankah ada saatnya orang-orang harus diajari bagaimana mencapai orgasme?' dia pasti tertawa. Karena hewan mencapainya tanpa pengajaran apapun, tidak ada Master, tidak ada Johnsons yang dibutuhkan, tidak ada Kinsey Report yang dibutuhkan.Hewan hanya mencintai - cinta terjadi secara alami.

\english
Nature does not flow according to your rules; it has its own rules. You have simply to be with it and it starts functioning. The day is not very far away when we will have to teach people how to breathe. Right now you laugh about it, but if you went back and asked Epicurus, 'Will there be a time when people will have to be taught how to achieve orgasm?' he would have laughed. Because animals achieve it without any teaching, no Masters, no Johnsons are needed, no Kinsey Report is needed. Animals simply love - love happens naturally.

\bahasa
Sekarang ada klinik di Amerika Serikat dimana mereka mengajari orang bagaimana mencapai orgasme. Dan jika melalui pembelajaran dan latihan engkau mencapai orgasme, ingatlah dengan baik, itu bukan hal yang nyata. Karena kemudian engkau memanipulasii, kemudian engkau mengendalikannya, kemudian entah bagaimana engkau memaksanya, dan orgasme hanya terjadi dalam keadaan keihlasan - dan sebuah keikhlasan tidak dapat diajarkan.

\english
Now there are clinics in the United States where they teach people how to achieve orgasm. And if through learning and training you achieve orgasm, remember well, it is not the real thing. Because then you are manipulating it, then you are controlling it, then somehow you are forcing it, and orgasm happens only in a let-go - and a let-go cannot be taught.

\bahasa
Engkau tidak dapat mengajari orang bagaimana cara tertidur. Jika engkau mencoba untuk mengajari maka engkau akan mengganggu tidur mereka karena jika mereka mencoba - apapun - itu hanya akan menjadi gangguan. Engkau cukup tidur, engkau cukup meletakkan kepala di atas bantal dan pergi tidur. Jika engkau melakukan sesuatu, maka tindakan itu akan menjadi kehancuran. Hidup itu seperti tidur; hidup itu seperti bernafas.

\english
You cannot teach people how to go to sleep. If you try to teach then you will disturb their sleep because if they try - anything - it will only be a disturbance. You simply go to sleep, you simply put your head on the pillow and go to sleep. If you do something, then that very doing will be the undoing. Life is just like sleep; life is just like breathing.

\bahasa
TIDAK ADA LATIHAN YANG DIBUTUHKAN, PIKIRANNYA SANGAT SEDERHANA DAN TIDAK MENGENAL HAMBATAN.


\english
NO APPLICATION WAS NEEDED - HIS MIND WAS PERFECTLY SIMPLE AND KNEW NO OBSTACLE.

\bahasa
Bila pikiranmu jernih pikiran itu memiliki kejelasan, engkau tidak perlu mengikuti peraturan apapun. Engkau tidak perlu membawa kitab suci apapun di kepala - engkau cukup melihat. Semuanya transparan, karena engkau jernih.

\english
When your mind is clear it has a clarity, you need not follow any rules. You need not carry any scriptures in the head - you simply look. Everything is transparent, because you are clear.

\bahasa
JADI, SAAT SEPATUNYA PAS, KAKI DILUPAKAN, SAAT SABUK PAS, PERUT DILUPAKAN, SAAT HATI BENAR, 'UNTUK' DAN 'TERHADAP' DILUPAKAN.

\english
SO, WHEN THE SHOE FITS, THE FOOT IS FORGOTTEN, WHEN THE BELT FITS, THE BELLY IS FORGOTTEN, WHEN THE HEART IS RIGHT, 'FOR' AND 'AGAINST' ARE FORGOTTEN.

\bahasa
Ingat, ini adalah salah satu mantra terbesar: Bila sepatunya pas, kaki dilupakan.

\english
Remember, this is one of the greatest mantras: When the shoe fits, the foot is forgotten.

\bahasa
Kapan pun engkau sehat engkau tidak tahu apa-apa tentang tubuhmu - tubuh dilupakan. Bila ada beberapa penyakit, baru kemudian engkau tidak dapat melupakan tubuh. Apakah engkau tahu jika ada kepala tanpa sakit kepala? Bila ada sakit kepala engkau tidak dapat melupakan kepala. Saat sepatu itu menjepit maka sepatu itu tidak pas. Bila engkau tidak memiliki sakit kepala, di manakah kepala itu? engkau benar-benar melupakannya. Apa pun yang sehat dilupakan tapi apa pun yang sakit diingat - ini menjadi catatan terus-menerus dalam pikiran, ketegangan terus-menerus dalam pikiran.

\english
Whenever you are healthy you don't know anything about your body - the body is forgotten. When there is some illness, only then can you not forget the body. Do you know if there is any head without a headache? When there is a headache you cannot forget the head. When the shoe is pinching then it doesn’t fit. When you don’t have any headache, where is the head? You completely forget about it. Whatsoever is healthy is forgotten but whatsoever is ill is remembered - it becomes a continuous note in the mind, a continuous tension in the mind.

\bahasa
Seorang manusia sempurna didalam Tao tidak mengenal dirinya sendiri; ENGKAU tahu, karena engkausedang sakit. Ego adalah penyakit, penyakit yang parah, karena engkau harus selalu mengingat bahwa engkau adalah seseorang. Ini menunjukkan bahwa Anda berada dalam 'ketidaknyamanan' yang dalam. Penyakit menciptakan ego; seseorang keberadaan yang sempurna sehat alami benar-benar melupakan. Dia seperti awan, seperti angin sepoi-sepoi, seperti batu, seperti pohon, seperti burung - tapi tidak pernah seperti manusia. Dia tidak ada, karena hanya penyakit, seperti luka, harus diingat.

\bahasa
Mengingat adalah sebuah mekanisme untuk keselamatan dan keamanan: jika ada duri di kakimu, engkau harus mengingat. Pikiran akan terus berlanjut berulang kali menunjuk titik itu karena duri itu harus dilempar keluar. Jika engkau melupakannya, duri itu akan tetap berada di sana dan itu akan menjadi berbahaya; itu bisa meracuni seluruh tubuh. Bila ada sakit kepala, tubuh menyuruhmu mengingatnya, ada sesuatu yang harus dilakukan. Jika engkau lupa, sakit kepala dapat menjadi berbahaya.

\english
Remembering is a mechanism for safety and security: if there is a thorn in your foot, you have to remember. The mind will go continuously again and again to the spot because the thorn has to be thrown out. If you forget it, the thorn will remain there and it will become dangerous; it may poison the whole body. When there is a headache the body tells you to remember it, something has to be done. If you forget it, the headache may become dangerous.

\bahasa
Tubuh memberitahumu kapan pun ada beberapa penyakit, ada sesuatu yang salah - hal ini menarik perhatianmu. Tapi saat tubuh sehat, engkau melupakannya; engkau menjadi 'tanpa-tubuh' saat tubuh sehat. Dan inilah satu-satunya definisi kesehatan: kesehatan adalah bila tidak ada kesadaran tubuh. Jika ada semacam kesadaran tubuh, maka bagian itu tidak sehat.

\english
The body tells you whenever there is some illness, something wrong - it attracts your attention. But when the body is healthy, you forget it; you become 'body-less' when the body is healthy. And this is the only definition of health: health is when there is no consciousness of the body. If there is any sort of consciousness of the body, then that part is not healthy.

\bahasa
Hal yang sama berlaku untuk pikiran. Bila kesadaranmu sehat, tidak ada ego - engkau tidak tahu apa-apa tentang dirimu sendiri. Engkau tidak terus mengingatkan diri sendiri bahwa 'aku adalah sesuatu', engkau hanya rileks. Engkau ada, tapi tidak ada 'aku'. Itu hanya 'ke-aku-annya' , kebeginiannya', tapi tidak ada 'aku', tidak ada ego yang mengkristal. Diri tidak ada.

\english
The same applies to the mind. When your consciousness is healthy, there is no ego - you don't know anything about yourself. You don't go on reminding yourself that 'I am something,' you simply relax. You are, but there is no 'I'. It is a simple 'am-ness', an 'is-ness', but there is no 'I', no crystallised ego. The self is not there.

\bahasa
JADI, SAAT SEPATUNYA PAS, KAKI DILUPAKAN, SAAT SABUK PAS, PERUT DILUPAKAN, SAAT HATI BENAR, 'UNTUK' DAN 'TERHADAP' DILUPAKAN.

\english
SO, WHEN THE SHOE FITS, THE FOOT IS FORGOTTEN, WHEN THE BELT FITS, THE BELLY IS FORGOTTEN, WHEN THE HEART IS RIGHT, 'FOR' AND 'AGAINST' ARE FORGOTTEN.

\bahasa
Inilah salah satu hal terdalam yang harus dipahami. Bila hati benar, semua 'untuk' dan 'terhadap' dilupakan. Bila hati salah, sakit, maka engkau terus-menerus dibebani, khawatir: ini benar dan itu salah - dan yang benar harus diikuti, yang salah harus dihindari. Seluruh kehidupan menjadi sebuah pergumulan bagaimana menghindari yang salah dan bagaimana mencapai yang benar. Dan ini bukan cara untuk mencapai yang benar! Inilah cara untuk melewatkannya selamanya.

\english
This is one of the deepest things to be understood. When the heart is right, all 'for' and 'against' are forgotten. When the heart is wrong, ill, then you go on continuously being burdened, worried: this is right and that is wrong - and the right should be followed, the wrong should be avoided. The whole of life becomes a struggle how to avoid the wrong and how to achieve the right. And this is not the way to achieve the right! This is the way to miss it forever.

\bahasa
Lihatlah .... engkau memiliki kemarahan, seks, keserakahan. Jika engkau mengatakan bahwa kemarahan adalah salah maka seluruh hidupmu akan berlalu dalam keadaan marah. Terkadang engkau akan marah, dan terkadang engkau akan marah karena kemarahanmu- itu akan menjadi satu-satunya perbedaan. Terkadang engkau akan marah, dan saat kemarahan hilang engkau akan marah karena kemarahan; engkau memanggil ini sebagai pertobatan. Dan kemudian engkau akan memutuskan untuk tidak marah lagi, tapi engkau akan marah lagi, karena kedua kedaaan tersebut adalah kemarahan. Terkadang engkau marah terhadap orang lain, terkadang engkau marah terhadap diri sendiri karena engkau marah.

\english
Look.... You have anger, sex, greed. If you say anger is wrong then your whole life will be passed in an angry state. Sometimes you will be angry, and sometimes you will be angry because of your anger - that will be the only difference. Sometimes you will be angry, and when the anger is gone you will be angry because of the anger; you call this repentance. And then you will decide never to be angry again, but you will be angry again, because both states are anger. Sometimes you are angry against someone else, sometimes you are angry against yourself because you were angry.


