\chapter{Ketika Sepatu Pas}

\bahasa
11 Oktober 1974 pagi di Aula Buddha

\english
11 October 1974 am in Buddha Hall

\bahasa
Chu'i si juru gambar dapat menggambar lingkaran lebih sempurna dengan tangan hampa daripada dengan sebuah kompas.

Jemarinya melahirkan bentuk spontan entah dari mana. Pikirannya keji selagi bebas dan tanpa mempedulikan apa yang sedang dilakukannya.

Tidak ada latihan yang dibutuhkan, pikirannya sangat sederhana dan tidak mengenal hambatan.

Jadi, saat sepatunya pas, kaki dilupakan, saat sabuk pas, perut dilupakan, saat hati benar, 'untuk' dan 'terhadap' dilupakan.

Tidak ada penggerak, tidak ada dorongan, tidak ada kebutuhan, tidak ada tarikan: maka usahamu terkendali. Engkau adalah orang yang bebas.

Santai adalah benar. Mulailah dengan benar dan engkau santai. Lanjutkan dengan santai dan engkau benar. Cara yang tepat untuk menjadi santai adalah melupakan jalan yang benar dan melupakan bahwa jalannya santai.

\english
CHU'I THE DRAFTSMAN COULD DRAW MORE PERFECT CIRCLES FREEHAND THAN WITH A COMPASS.

HIS FINGERS BROUGHT FORTH SPONTANEOUS FORMS FROM NOWHERE. HIS MIND WAS MEANWHILE FREE AND WITHOUT CONCERN WITH WHAT HE WAS DOING.

NO APPLICATION WAS NEEDED, HIS MIND WAS PERFECTLY SIMPLE AND KNEW NO OBSTACLE.

SO, WHEN THE SHOE FITS, THE FOOT IS FORGOTTEN, WHEN THE BELT FITS, THE BELLY IS FORGOTTEN, WHEN THE HEART IS RIGHT, ’FOR’ AND ’AGAINST’ ARE FORGOTTEN.

NO DRIVES, NO COMPULSIONS, NO NEEDS, NO ATTRACTIONS: THEN YOUR AFFAIRS ARE UNDER CONTROL. YOU ARE A FREE MAN.

EASY IS RIGHT. BEGIN RIGHT AND YOU ARE EASY. CONTINUE EASY AND YOU ARE RIGHT. THE RIGHT WAY TO GO EASY IS TO FORGET THE RIGHT WAY AND FORGET THAT THE GOING IS EASY.

\bahasa
Chuang Tzu adalah salah satu bunga yang langka, langka bahkan dari Buddha atau Yesus. Karena Buddha dan Yesus menekankan usaha dan Chuang Tzu menekankan tanpa usaha.

\english
Chuang Tzu is one of the rarest of flowerings, rarer even than a Buddha or a Jesus. Because Buddha and Jesus emphasise effort and Chuang Tzu emphasises effortlessness.

\bahasa
Banyak yang dapat dilakukan melalui usaha tapi lebih banyak dapat dilakukan melalui tanpa usaha. Banyak yang bisa dicapai melalui kehendak tapi banyak lagi bisa dicapai melalui tanpa kehendak. Dan apa pun yang engkau capai melalui kehendak akan selalu menjadi beban bagimu; itu akan selalu menjadi sebuah konflik, sebuah ketegangan batin, dan engkau dapat kehilangannya kapanpun. Hal itu harus terus dipertahankan - dan mempertahankannya membutuhkan energi, mempertahankannya akhirnya mencerai-beraikan dirimu.

\english
Much can be done through effort but more can be done through effortlessness. Much can be achieved through will but much more can be achieved through will-lessness. And whatsoever you achieve through will will always remain a burden to you; it will always be a conflict, an inner tension, and you can lose it at any moment.It has to be maintained continuously – and maintaining it takes energy, maintaining it finally dissipates you.

\bahasa
Hanya apa yang dicapai melalui tanpa-usaha tidak akan pernah menjadi sebuah beban bagimu, dan hanya apa yang tidak menjadi sebuah beban dapat abadi. Hanya apa yang tidak dengan cara apa pun yang tidak wajar dapat tetap bersamamu selamanya dan selamanya.

\english
Only that which is attained through effortlessness will never be a burden to you, and only that which is not a burden can be eternal. Only that which is not in any way unnatural can remain with you forever and forever.

\bahasa
Chuang Tzu mengatakan bahwa yang sebenarnya, yang ilahi, eksistensial, dicapai dengan kehilangan diri sepenuhnya di dalamnya. Bahkan upaya untuk mencapainya menjadi penghalang - maka engkau tidak dapat kehilangan diri sendiri. Bahkan usaha untuk kehilangan diri menjadi penghalang.

\english
Chuang Tzu says that the real, the divine, the existential, is to be attained by losing yourself completely in it. Even the effort to attain it becomes a barrier – then you cannot lose yourself. Even the effort to lose yourself becomes a barrier.
