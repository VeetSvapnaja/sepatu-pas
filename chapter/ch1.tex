\chapter{Ketika Sepatu Pas}

\bahasa
11 Oktober 1974 pagi di Aula Buddha

\english
11 October 1974 am in Buddha Hall

\bahasa
CHU'I SI JURU GAMBAR DAPAT MENGGAMBAR LINGKARAN LEBIH SEMPURNA DENGAN TANGAN HAMPA DARIPADA DENGAN SEBUAH KOMPAS.

\english
CHU'I THE DRAFTSMAN COULD DRAW MORE PERFECT CIRCLES FREEHAND THAN WITH A COMPASS.

\bahasa
JEMARINYA MELAHIRKAN BENTUK SPONTAN ENTAH DARI MANA. PIKIRANNYA LIAR SELAGI BEBAS DAN TANPA MEMPEDULIKAN APA YANG SEDANG DILAKUKANNYA.

\english
HIS FINGERS BROUGHT FORTH SPONTANEOUS FORMS FROM NOWHERE. HIS MIND WAS MEANWHILE FREE AND WITHOUT CONCERN WITH WHAT HE WAS DOING.

\bahasa
TIDAK ADA LATIHAN YANG DIBUTUHKAN, PIKIRANNYA SANGAT SEDERHANA DAN TIDAK MENGENAL HAMBATAN.

\english
NO APPLICATION WAS NEEDED, HIS MIND WAS PERFECTLY SIMPLE AND KNEW NO OBSTACLE.

\bahasa
JADI, SAAT SEPATUNYA PAS, KAKI DILUPAKAN, SAAT SABUK PAS, PERUT DILUPAKAN, SAAT HATI BENAR, 'UNTUK' DAN 'TERHADAP' DILUPAKAN.

\english
SO, WHEN THE SHOE FITS, THE FOOT IS FORGOTTEN, WHEN THE BELT FITS, THE BELLY IS FORGOTTEN, WHEN THE HEART IS RIGHT, 'FOR' AND 'AGAINST' ARE FORGOTTEN.

\bahasa
TIDAK ADA PENGGERAK, TIDAK ADA DORONGAN, TIDAK ADA KEBUTUHAN, TIDAK ADA TARIKAN: MAKA USAHAMU TERKENDALI. ENGKAU ADALAH ORANG YANG BEBAS.

\english
NO DRIVES, NO COMPULSIONS, NO NEEDS, NO ATTRACTIONS: THEN YOUR AFFAIRS ARE UNDER CONTROL. YOU ARE A FREE MAN.

\bahasa
SANTAI ADALAH BENAR. MULAILAH DENGAN BENAR DAN ENGKAU SANTAI. LANJUTKAN DENGAN SANTAI DAN ENGKAU BENAR. CARA YANG TEPAT UNTUK MENJADI SANTAI ADALAH MELUPAKAN JALAN YANG BENAR DAN MELUPAKAN BAHWA JALANNYA SANTAI.

\english
EASY IS RIGHT. BEGIN RIGHT AND YOU ARE EASY. CONTINUE EASY AND YOU ARE RIGHT. THE RIGHT WAY TO GO EASY IS TO FORGET THE RIGHT WAY AND FORGET THAT THE GOING IS EASY.

\bahasa
Chuang Tzu adalah salah satu bunga yang langka, lebih langka bahkan daripada seorang Buddha atau seorang Yesus. Karena Buddha dan Yesus menekankan usaha dan Chuang Tzu menekankan tanpa usaha.

\english
Chuang Tzu is one of the rarest of flowerings, rarer even than a Buddha or a Jesus. Because Buddha and Jesus emphasise effort and Chuang Tzu emphasises effortlessness.

\bahasa
Banyak yang dapat dilakukan melalui usaha tapi lebih banyak dapat dilakukan melalui tanpa usaha. Banyak yang dapat dicapai melalui kehendak tapi banyak lagi bisa dicapai melalui tanpa kehendak. Dan apa pun yang engkau capai melalui kehendak akan selalu menjadi beban bagimu; itu akan selalu menjadi sebuah konflik, sebuah ketegangan batin, dan engkau dapat kehilangannya kapanpun. Hal itu harus terus dipertahankan - dan mempertahankannya membutuhkan energi, mempertahankannya akhirnya mencerai-beraikan dirimu.

\english
Much can be done through effort but more can be done through effortlessness. Much can be achieved through will but much more can be achieved through will-lessness. And whatsoever you achieve through will will always remain a burden to you; it will always be a conflict, an inner tension, and you can lose it at any moment.It has to be maintained continuously - and maintaining it takes energy, maintaining it finally dissipates you.

\bahasa
Hanya apa yang dicapai melalui tanpa-usaha tidak akan pernah menjadi sebuah beban bagimu, dan hanya apa yang tidak menjadi sebuah beban dapat abadi. Hanya apa yang tidak dengan cara apa pun yang tidak wajar dapat tetap bersamamu selamanya dan selamanya.

\english
Only that which is attained through effortlessness will never be a burden to you, and only that which is not a burden can be eternal. Only that which is not in any way unnatural can remain with you forever and forever.

\bahasa
Chuang Tzu mengatakan bahwa yang sebenarnya, yang ilahi, eksistensial, dicapai dengan kehilangan diri sepenuhnya di dalamnya. Bahkan upaya untuk mencapainya menjadi sebuah penghalang - maka engkau tidak dapat kehilangan diri sendiri. Bahkan usaha untuk kehilangan diri menjadi sebuah penghalang.

\english
Chuang Tzu says that the real, the divine, the existential, is to be attained by losing yourself completely in it. Even the effort to attain it becomes a barrier - then you cannot lose yourself. Even the effort to lose yourself becomes a barrier.

\bahasa
Bagaimana engkau dapat membuat usaha untuk kehilangan diri sendiri? Semua usaha lahir dari ego, dan melalui usaha ego diperkuat. Ego adalah penyakitnya. Jadi semua usaha harus ditinggalkan sama sekali, tidak ada yang harus dilakukan; kita harus kehilangan diri sepenuhnya dalam eksistensial. Kita harus menjadi seperti anak kecil, baru lahir, tidak tahu apa yang benar, tidak tahu apa yang salah, tidak mengetahui perbedaan apapun. Begitu perbedaan masuk, setelah engkau tahu ini benar dan itu salah, engkau sudah sakit, dan engkau jauh dari kenyataan.

\english
How can you make any effort to lose yourself? All effort is born out of the ego, and through effort ego is strengthened. Ego is the disease. So all effort has to be left completely, nothing is to be done; one has to lose oneself completely in the existential. One has to become again like a small child, just born, not knowing what is right, not knowing what is wrong, not knowing any distinctions. Once distinctions enter, once you know this is right and that is wrong, you are already ill, and you are far away from reality.

\bahasa
Seorang anak hidup secara alami - dia total. Dia tidak membuat usaha apapun, karena melakukan usaha berarti engkau berkelahi dengan diri sendiri. Sebagian dari dirimu adalah untuk dan sebagian dari dirimu melawan - maka dari itu adanya upaya.

\english
A child lives naturally - he is total. He does not make any effort, because making an effort means you are fighting with yourself. A part of you is for and a part of you is against - hence the effort.

\bahasa
Engkau dapat meraih banyak, ingatlah. Di dunia ini, khususnya, engkau dapat meraih banyak melalui usaha karena usaha adalah agresi, usaha adalah kekerasan, usaha adalah persaingan. Tapi di dunia lain tidak ada yang dapat dicapai melalui usaha, dan mereka yang memulai dengan usaha akhirnya juga harus menjatuhkannya.

\english
You can achieve much, remember. In this world, particularly, you can achieve much through effort because effort is aggression, effort is violence, effort is competition. But in the other world nothing can be achieved through effort, and those who start with effort finally have also to drop it.

\bahasa
Buddha bekerja selama enam tahun, terus bermeditasi, berkonsentrasi - dia menjadi seorang pertapa. Dia melakukan semua yang dapat dilakukan oleh manusia, tidak ada satu pun batu yang terlewat - dia mempertaruhkan seluruh keberadaannya. Tapi itu adalah usaha, ego ada di sana, dan dia gagal.

\english
Buddha worked for six years, continuously meditating, concentrating - he became an ascetic. He did all that can be done by a human being, not a single stone was left unturned - he staked his whole being. But it was an effort, the ego was there, and he failed.

\bahasa
Tidak ada yang gagal seperti ego di yang terutama; tidak ada yang berhasil seperti ego di dunia ini.

\english
Nothing fails like the ego in the Ultimate; nothing succeeds like the ego in this world.

\bahasa
Dalam dunia materi tidak ada yang berhasil seperti ego; dalam dunia kesadaran tidak ada yang gagal seperti ego. Kasusnya justru berlawanan - dan memang harus seperti itu karena dimensinya justru berlawanan.

\english
In the world of matter nothing succeeds like the ego; in the world of consciousness nothing fails like the ego. The case is just the opposite - and it has to be so because the dimension is just the opposite.

\bahasa
Buddha gagal total. Setelah enam tahun dia benar-benar frustrasi, dan ketika aku mengatakan benar-benar, maksud ku benar-benar sepenuhnya. Tak ada satu pun serpihan harapan yang tersisa, dia menjadi sangat tidak berdaya. Dalam tanpa daya itu, dia menjatuhkan semua usaha. Dia sudah menjatuhkan dunia, dia sudah meninggalkan kerajaannya; semua milik dunia yang terlihat ini, dia telah meninggalkannya, melepaskan semuanya.

\english
Buddha failed absolutely. After six years he was completely frustrated, and when I say completely, I mean completely. Not even a single fragment of hope remained, he became absolutely hopeless. In that hopelessness he dropped all effort. He had already dropped the world, he had already left his kingdom; all that belonged to this visible world, he had left, renounced.

\bahasa
Sekarang setelah enam tahun usaha keras ia meninggalkan juga semua milik dunia lain. Dia benar-benar hampa - kosong. Malam itu ia tidur dengan kualitas tidur yang berbeda karena tidak ada ego; kualitas keheningan yang berbeda muncul karena tidak ada usaha; kualitas yang berbeda terjadi padanya malam itu karena tidak ada mimpi.

\english
Now after six years of strenuous effort he left also all that belonged to the other world. He was in a complete vacuum - empty. That night he slept a different quality of sleep because there was no ego; a different quality of silence arose because there was no effort; a different quality of being happened to him that night because there was no dreaming.

\bahasa
Jika tidak ada usaha, tidak ada yang tidak lengkap, maka tidak perlu bermimpi. Mimpi selalu menyelesaikan sesuatu: sesuatu yang tidak lengkap pada siang hari akan selesai dalam mimpi karena pikiran memiliki kecenderungan untuk menyelesaikan semuanya. Jika tidak lengkap maka pikiran akan selalu tidak nyaman. Usaha dimasukkan ke dalam banyak hal dan jika hal-hal itu tetap tidak lengkap, sebuah mimpi dibutuhkan.

\english
If there is no effort, nothing is incomplete, then there is no need to dream. A dream is always to complete something: something which has remained incomplete in the day will be completed in a dream because mind has a tendency to complete everything. If it is not complete then the mind will always be uneasy. Effort is put into many things and if they remain incomplete, a dream is needed.

\bahasa
Bila ada keinginan, pasti ada mimpi, karena menginginkan adalah bermimpi - bermimpi hanyalah bayangan keinginan.

\english
When there is desire, there is bound to be dreaming, because desiring is dreaming – dreaming is just a shadow of desiring.

\bahasa
Malam itu, ketika tidak ada yang dapat dilakukan - dunia ini sudah tidak ada gunanya, sekarang dunia lain juga tidak berguna - semua motivasi untuk bergerak berhenti. Tidak ada tempat untuk pergi, dan tidak ada seseorang yang dapat pergi ke mana pun. Tidur malam itu menjadi samadhi, itu menjadi satori; hal itu menjadi hal paling utama yang dapat terjadi pada seseorang. Buddha berbunga malam itu dan di pagi hari dia tercerahkan. Dia membuka matanya, melihat bintang terakhir menghilang di langit, dan semuanya ada di sana. Itu selalu ada di sana, tapi dia sangat menginginkannya sehingga dia tidak dapat melihatnya. Itu selalu ada di sana; tapi dia telah bergerak jauh di masa depan dengan keinginan sehingga dia tidak dapat melihat pada disini dan sekarang.

\english
That night, when there was nothing to be done - this world was already useless, now the other world was also useless - all motivation to move ceased. There was nowhere to go, and there was no one to go anywhere That night sleep became samadhi, it became satori; it became the ultimate thing that can happen to a man. Buddha flowered that night and in the morning he was enlightened. He opened his eyes, looked at the last star disappearing in the sky, and everything was there. It had always been there, but he had wanted it so much he couldn't see it. It had always been there; but he had been moving so much in the future with the desire that he could not look at the here and now.

\bahasa
Malam itu tidak ada keinginan, tidak ada tujuan, tidak ada tempat untuk pergi, dan tidak ada seseorang yang dapat pergi ke mana pun - semua usaha berhenti. Tiba-tiba dia menjadi sadar akan dirinya sendiri, tiba-tiba dia menyadari kenyataan seperti apa adanya. Chuang Tzu mengatakan sejak awal: Jangan melakukan usaha apapun. Dan dia benar. Karena engkau tidak akan pernah membuat usaha total seperti Buddha. Engkau tidak akan pernah merasa frustrasi sehingga usaha itu jatuh dengan sendirinya; itu akan selalu tidak lengkap. Dan pikiranmu akan selalu terus berkata: Sedikit lagi dan sesuatu akan terjadi, selangkah lagi. Tujuannya sudah dekat, mengapa engkau merasa berkecil hati? Hanya sedikit lagi usaha diperlukan karena tujuan semakin dekat setiap hari.

\english
That night there was no desire, no goal, nowhere to go, and no one to go anywhere – all effort ceased. Suddenly he became aware of himself, suddenly he became aware of the reality as it is. Chuang Tzu says from the very beginning: Don't make any effort. And he is right. Because you will never make such a total effort as Buddha. You will never be so frustrated that the effort drops by itself; it will always be incomplete. And your mind will always go on saying: A little more and something will happen, a step more. The goal is near, why are you getting dejected? Just a little more effort is needed because the goal is coming nearer every day.

\bahasa
Karena engkau tidak akan pernah berusaha sekuat tenaga sehingga engkau tidak akan pernah benar-benar putus asa. Dan engkau dapat melanjutkan usaha setengah hati ini untuk banyak kehidupan - itulah yang telah engkau lakukan di masa lalu. Engkau tidak berada di sini untuk pertama kalinya dihadapanku. Engkau tidak di sini untuk pertama kalinya membuat beberapa upaya untuk menyadari yang benar, yang sebenarnya. Engkau telah berkali-kali melakukannya, berkali-kali, satu juta kali di masa lalu - tapi engkau masih berharap.

\english
Because you will never make so absolute an effort you will never be completely hopeless. And you can continue this half-hearted effort for many lives - that is what you have been doing in the past. You are not here for the first time before me. You are not here for the first time making some effort to realise the true, the real. You have done it many, many times, a million times in the past - but you are still hopeful.

\bahasa
Chuang Tzu mengatakan; lebih baik menjatuhkan usaha di awal. Itu harus dijatuhkan: entah engkau menjatuhkannya di awal atau engkau harus menjatuhkannya di akhir. Tapi akhirnya mungkin tidak segera datang! Jadi ada dua cara: baik melakukan usaha total ... jadi total sehingga semua harapan hancur dan engkau menyadari bahwa tidak ada yang dapat dicapai melalui usaha, bahkan tidak ada satu serpihan kecil di suatu tempat di alam bawah sadar yang masih tersisa dan mengatakan : Lakukan sedikit lagi dan ini akan tercapai .... baik itu membuat usaha total, maka itu jatuh dengan sendirinya, atau sama sekali tidak berusaha. Hanya pahami semuanya. Jangan pindah ke sana bagaimanapun juga. Ingat satu hal ... engkau tidak dapat keluar dari situ jika tidak lengkap; begitu masuk, itu harus selesai. Karena pikiran memiliki kecenderungan untuk menyelesaikan segalanya - tidak hanya pikiran manusia, bahkan pikiran binatang. Jika engkau menggambar setengah lingkaran, tidak lengkap, dan seekor gorila datang dan melihatnya, dan jika ada kapur tulis di sana, dia akan segera menyelesaikannya.

\english
Chuang Tzu says; it is better to drop effort in the beginning. It has to be dropped: either you drop it in the beginning or you will have to drop it in the end. But the end may not come soon! So there are two ways: either make a total effort... so total that all hope is shattered and you come to realise that nothing can be achieved through effort, there is not even a single small fragment somewhere in the unconscious still lingering and saying: Do a little more and this will be achieved.... either make a total effort, then it drops by itself, or don't make any effort at all. Just understand the whole thing. Don't move into it at all. Remember one thing... you cannot come out of it if it is incomplete; once entered, it has to be completed. Because the mind has a tendency to complete everything - not only the human mind, even the animal mind. If you draw a half circle, incomplete, and a gorilla comes and sees it, and if some chalk is there, he will immediately complete it.

\bahasa
Pikiranmu memiliki kecenderungan untuk menyelesaikan - apapun yang tidak lengkap memberimu ketegangan. Jika engkau ingin tertawa dan engkau tidak bisa, akan ada ketegangan. Jika engkau ingin menangis dan tidak bisa, akan ada ketegangan. Jika engkau ingin marah dan tidak bisa, akan ada ketegangan. Itu sebabnya engkau telah sakit begitu lama; semuanya telah ditinggalkan tidak lengkap!

\english
Your mind as such has a tendency to complete - anything incomplete gives you tension. If you wanted to laugh and you could not, there will be tension. If you wanted to cry and could not, there will be tension. If you wanted to be angry and could not, there will be tension. That's why you have been ill for so long; everything has been left incomplete!
