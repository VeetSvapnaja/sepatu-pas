\chapter{Ketika Sepatu Pas}

\bahasa
11 Oktober 1974 pagi di Aula Buddha

\english
11 October 1974 am in Buddha Hall

\bahasa
CHU'I SI JURU GAMBAR DAPAT MENGGAMBAR LINGKARAN LEBIH SEMPURNA DENGAN TANGAN HAMPA DARIPADA DENGAN SEBUAH KOMPAS.

\english
CHU'I THE DRAFTSMAN COULD DRAW MORE PERFECT CIRCLES FREEHAND THAN WITH A COMPASS.

\bahasa
JEMARINYA MELAHIRKAN BENTUK SPONTAN ENTAH DARI MANA. PIKIRANNYA LIAR SELAGI BEBAS DAN TANPA MEMPEDULIKAN APA YANG SEDANG DILAKUKANNYA.

\english
HIS FINGERS BROUGHT FORTH SPONTANEOUS FORMS FROM NOWHERE. HIS MIND WAS MEANWHILE FREE AND WITHOUT CONCERN WITH WHAT HE WAS DOING.

\bahasa
TIDAK ADA LATIHAN YANG DIBUTUHKAN, PIKIRANNYA SANGAT SEDERHANA DAN TIDAK MENGENAL HAMBATAN.

\english
NO APPLICATION WAS NEEDED, HIS MIND WAS PERFECTLY SIMPLE AND KNEW NO OBSTACLE.

\bahasa
JADI, SAAT SEPATUNYA PAS, KAKI DILUPAKAN, SAAT SABUK PAS, PERUT DILUPAKAN, SAAT HATI BENAR, 'UNTUK' DAN 'TERHADAP' DILUPAKAN.

\english
SO, WHEN THE SHOE FITS, THE FOOT IS FORGOTTEN, WHEN THE BELT FITS, THE BELLY IS FORGOTTEN, WHEN THE HEART IS RIGHT, ’FOR’ AND ’AGAINST’ ARE FORGOTTEN.

\bahasa
TIDAK ADA PENGGERAK, TIDAK ADA DORONGAN, TIDAK ADA KEBUTUHAN, TIDAK ADA TARIKAN: MAKA USAHAMU TERKENDALI. ENGKAU ADALAH ORANG YANG BEBAS.

\english
NO DRIVES, NO COMPULSIONS, NO NEEDS, NO ATTRACTIONS: THEN YOUR AFFAIRS ARE UNDER CONTROL. YOU ARE A FREE MAN.

\bahasa
SANTAI ADALAH BENAR. MULAILAH DENGAN BENAR DAN ENGKAU SANTAI. LANJUTKAN DENGAN SANTAI DAN ENGKAU BENAR. CARA YANG TEPAT UNTUK MENJADI SANTAI ADALAH MELUPAKAN JALAN YANG BENAR DAN MELUPAKAN BAHWA JALANNYA SANTAI.

\english
EASY IS RIGHT. BEGIN RIGHT AND YOU ARE EASY. CONTINUE EASY AND YOU ARE RIGHT. THE RIGHT WAY TO GO EASY IS TO FORGET THE RIGHT WAY AND FORGET THAT THE GOING IS EASY.

\bahasa
Chuang Tzu adalah salah satu bunga yang langka, langka bahkan daripada seorang Buddha atau seorang Yesus. Karena Buddha dan Yesus menekankan usaha dan Chuang Tzu menekankan tanpa usaha.

\english
Chuang Tzu is one of the rarest of flowerings, rarer even than a Buddha or a Jesus. Because Buddha and Jesus emphasise effort and Chuang Tzu emphasises effortlessness.

\bahasa
Banyak yang dapat dilakukan melalui usaha tapi lebih banyak dapat dilakukan melalui tanpa usaha. Banyak yang bisa dicapai melalui kehendak tapi banyak lagi bisa dicapai melalui tanpa kehendak. Dan apa pun yang engkau capai melalui kehendak akan selalu menjadi beban bagimu; itu akan selalu menjadi sebuah konflik, sebuah ketegangan batin, dan engkau dapat kehilangannya kapanpun. Hal itu harus terus dipertahankan - dan mempertahankannya membutuhkan energi, mempertahankannya akhirnya mencerai-beraikan dirimu.

\english
Much can be done through effort but more can be done through effortlessness. Much can be achieved through will but much more can be achieved through will-lessness. And whatsoever you achieve through will will always remain a burden to you; it will always be a conflict, an inner tension, and you can lose it at any moment.It has to be maintained continuously – and maintaining it takes energy, maintaining it finally dissipates you.

\bahasa
Hanya apa yang dicapai melalui tanpa-usaha tidak akan pernah menjadi sebuah beban bagimu, dan hanya apa yang tidak menjadi sebuah beban dapat abadi. Hanya apa yang tidak dengan cara apa pun yang tidak wajar dapat tetap bersamamu selamanya dan selamanya.

\english
Only that which is attained through effortlessness will never be a burden to you, and only that which is not a burden can be eternal. Only that which is not in any way unnatural can remain with you forever and forever.

\bahasa
Chuang Tzu mengatakan bahwa yang sebenarnya, yang ilahi, eksistensial, dicapai dengan kehilangan diri sepenuhnya di dalamnya. Bahkan upaya untuk mencapainya menjadi sebuah penghalang - maka engkau tidak dapat kehilangan diri sendiri. Bahkan usaha untuk kehilangan diri menjadi sebuah penghalang.

\english
Chuang Tzu says that the real, the divine, the existential, is to be attained by losing yourself completely in it. Even the effort to attain it becomes a barrier – then you cannot lose yourself. Even the effort to lose yourself becomes a barrier.

\bahasa
Bagaimana engkau dapat membuat usaha untuk kehilangan diri sendiri? Semua usaha lahir dari ego, dan melalui usaha ego diperkuat. Ego adalah penyakitnya. Jadi semua usaha harus ditinggalkan sama sekali, tidak ada yang harus dilakukan; kita harus kehilangan diri sepenuhnya dalam eksistensial. Kita harus menjadi seperti anak kecil, baru lahir, tidak tahu apa yang benar, tidak tahu apa yang salah, tidak mengetahui perbedaan apapun. Begitu perbedaan masuk, setelah engkau tahu ini benar dan itu salah, engkau sudah sakit, dan engkau jauh dari kenyataan.

\english
How can you make any effort to lose yourself? All effort is born out of the ego, and through effort ego is strengthened. Ego is the disease. So all effort has to be left completely, nothing is to be done; one has to lose oneself completely in the existential. One has to become again like a small child, just born, not knowing what is right, not knowing what is wrong, not knowing any distinctions. Once distinctions enter, once you know this is right and that is wrong, you are already ill, and you are far away from reality.

\bahasa
Seorang anak hidup secara alami - dia total. Dia tidak membuat upaya apapun, karena melakukan upaya berarti engkau berkelahi dengan diri sendiri. Sebagian dari dirimu adalah untuk dan sebagian dari dirimu melawan - maka dari itu adanya upaya.

\english
A child lives naturally – he is total. He does not make any effort, because making an effort means you are fighting with yourself. A part of you is for and a part of you is against – hence the effort.

\bahasa
Engkau dapat meraih banyak, ingat. Di dunia ini, khususnya, engkau dapat meraih banyak melalui usaha karena usaha adalah agresi, usaha adalah kekerasan, usaha adalah persaingan. Tapi di dunia lain tidak ada yang dapat dicapai melalui usaha, dan mereka yang memulai dengan usaha akhirnya juga harus menjatuhkannya.

\english
You can achieve much, remember. In this world, particularly, you can achieve much through effort because effort is aggression, effort is violence, effort is competition. But in the other world nothing can be achieved through effort, and those who start with effort finally have also to drop it.


